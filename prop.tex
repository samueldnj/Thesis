\documentclass[12pt,]{scrartcl}
\usepackage[letterpaper, margin = 1in, footskip = 0.25in]{geometry}
\usepackage{amsmath,amssymb,setspace}
\usepackage{graphicx}
\usepackage{xcolor}
\doublespacing

\providecommand{\tightlist}{%
  \setlength{\itemsep}{0pt}\setlength{\parskip}{0pt}}

% Redefine \includegraphics so that, unless explicit options are
% given, the image width will not exceed the width of the page.
% Images get their normal width if they fit onto the page, but
% are scaled down if they would overflow the margins.
\makeatletter
\def\ScaleIfNeeded{%
  \ifdim\Gin@nat@width>\linewidth
    \linewidth
  \else
    \Gin@nat@width
  \fi
}
\makeatother
\let\Oldincludegraphics\includegraphics
{%
 \catcode`\@=11\relax%
 \gdef\includegraphics{\@ifnextchar[{\Oldincludegraphics}{\Oldincludegraphics[width=\ScaleIfNeeded]}}%
}%

\setcounter{secnumdepth}{0}

\ifxetex
  \usepackage[setpagesize=false, % page size defined by xetex
              unicode=false, % unicode breaks when used with xetex
              xetex]{hyperref}
\else
  \usepackage[unicode=true]{hyperref}
\fi
\hypersetup{breaklinks=true,
            bookmarks=true,
            pdfauthor={Samuel Johnson},
            pdftitle={Exploiting Interactions in Multispecies Fisheries to Assess and Avoid Constraining Species},
            colorlinks=true,
            citecolor=blue,
            urlcolor=blue,
            linkcolor=magenta,
            pdfborder={0 0 0}}
\urlstyle{same}  % don't use monospace font for urls


\title{Exploiting Interactions in Multispecies Fisheries to Assess and Avoid
Constraining Species}
\subtitle{Thesis Proposal}
\author{Samuel Johnson}
\date{\today}


\begin{document}
\maketitle

\abstract{Not Done Yet.}




\newpage

{
\hypersetup{linkcolor=black}
\setcounter{tocdepth}{3}
\tableofcontents
}
\section{Introduction}\label{introduction}

\subsection{Background}\label{background}

Sustainable management of any renewable resource requires understanding
the system dynamics in response to exploitation. In a multispecies
fisheries context the system is a collection of semi-discrete
self-sustaining fish populations or \emph{stocks} (Begg, Friedland, \&
Pearce (1999)), and the exploitation involves removing individuals by
fishing. Fishing effort impacts target species, non-target species and
fish habitat, and therefore a major challenge of multispecies fishery
management is to balance fishing yield with broader sustainability
goals.

Sustainable and scientifically defensible fishery management is built on
a foundation of fisheries stock assessment (Hilborn \& Walters (1992)).
Quantitative stock assessment methods combine elements of data science,
applied population ecology, risk assessment and resource management
(Figure 1). Analysts use data from multiple sources including scientific
surveys and commercial fishery monitoring to infer biological and
fishery dynamics and to characterise uncertainties and risks based on
these assessments. These inferences include estimates of species
abundance and productivity that are used to inform management decisions.

Stock assessments are lacking in most Canadian fisheries (Hutchings et
al. (2012)), especially for non-target species. One reason is that
non-target species are typically of lower commercial importance, so
there is limited interest in assessments. More commonly, data
limitations preclude the assessment of certain species, known as
data-limited species. Surveys designed for data-moderate target species
are often unsuitable for non-target species and leave managers with the
choice of conducting a flawed assessment, or no assessment at all.

A lack of assessments for some species within a multispecies fishery
threatens sustainable management of the whole fishery in two ways.
First, a lack of assessments creates conservation risks by weakening the
link between management decisions and stock status. The dynamic nature
of a fishery implies that the distribution of possible stock statuses
widens as time passes. Second, eco-certifiers typically require
up-to-date stock assessments for all species captured, regardless of
whether those stocks are targeted or not. A lack of eco-certification
reduces the capacity of a fishery for competition in international and
domestic markets, because buyers will prefer eco-certified products
(Pelc et al. (2015)).

\subsection{Assessments Acknowledging Technical
Interactions}\label{assessments-acknowledging-technical-interactions}

Stock assessments are traditionally performed for a single species at a
time, even though this approach may lead to sub-optimal outcomes for
multispecies fisheries (Sugihara et al. (1984); Gulland \& Garcia
(1984)). Sub-optimal outcomes may arise from not accounting for the
effects of interactions between species. Interactions between fishes in
multispecies fisheries are one of two types: ecological or technical.
Ecological interactions are either non-trophic, such as competition, or
trophic, between predator and prey. Ecological interactions affect
natural mortality of fish and may bias estimates of species productivity
when not taken into account (Mueter \& Megrey (2006)). Technical
interactions occur when multiple species are caught in the same
non-selective fishing gear, and are caused by multiple species of fish
being potentially available to the fishing gear.

Within the single-species paradigm, major stocks typically comprise
several distinct, but interacting, sub-stocks (Walters \& Martell
(2004); Ashleen J Benson, Cox, \& Cleary (2015)), such as Pacfic salmon
(\emph{Onchorynchys spp.}) (Simon \& Larkin (1972)). Multiple
ecologically and technically interacting populations (i.e., stocks) of
Chinook (\emph{O. tshawtcha}), Chum (\emph{O. keta}), Coho (\emph{O.
kisutch}), Pink (\emph{O. gorbausch}), Sockeye (\emph{O. nerka}) and
Steelhead (\emph{O. mykiss}) occur along Canada's Pacific coast. Each
species is made up of genetically distinct subpopulations, defined
mainly by discrete spawning habitats and run timing that establish
quasi-isolated reproductive populations (Ricker (1972)) connected by low
straying rates.

Managing hundreds of distinct fisheries is impractical (Walters \&
Martell (2004)) so salmon stocks are often grouped together into stock
complexes for management and assessment. For instance, in the Fraser
River, sub-populations of Chinook and Sockeye are grouped into aggregate
stock complexes called runs based on similarity in life history,
geographical locations of spawning habitat and arrival timing to
fisheries (English, Edgell, Bocking, Link, \& Raborn (2011); DFO
(1999)). Managing Pacific salmon in runs has both advantages and
disadvantages. Aggregation leads to increased management efficiency and
brings adds statistical benefits from data pooling. However, to avoid
overfishing due to different productivity levels of member stocks
(Figure 2), complexes must be managed according to the weakest stock's
productivity (Ricker (1958); Ricker (1973); Parkinson, Post, \& Cox
(2004)). The Late run of Fraser river Sockeye is managed for the weakest
stock: Cultus lake. The Cultus lake stock has historic abundances of up
to 700,000 spawners, but in 2004 fewer than 100 spawners returned from
the marine life phase. The decline of Cultus lake Sockeye is caused in
part by harvesting at average productivity for the complex (Team
(2009)). The Late run is now harvested according to the productivity of
the Cultus lake stock in order to avoid this effect on the declining
population.

The aggregate management schema used for Pacific salmon could be
modified and adopted in other multispecies fisheries. For example,
groundfish fisheries on the west coast of North America exploit stocks
of sablefish, Pacific halibut (\emph{Hippoglossus stenolopis}), several
species of rockfish (\emph{Sebastes spp.}), Pacific cod (\emph{Gadus
macrocephalus}), Dover sole (\emph{Microstomus pacificus}) and other
demersal species (Fisheries and Oceans, Canada (2015)). Different
groundfish genera and species have their own unique life histories and
reproductive strategies that respond differently to fishing pressure (S.
Jennings, Greenstreet, Reynolds, \& others (1999)). Different life
histories and reproductive strategies among groundfish imply different
productivity levels, similar to mixed-stock Pacific salmon fisheries.

Multiple interacting species with different productivity levels create
profitability constraints in multispecies fisheries managed through
quota systems (Hilborn, Punt, \& Orensanz (2004);Baudron \& Fernandes
(2015)). Constraints are caused by weaker, low productivity species that
cannot be avoided when targeting stronger, high productivity species.
Weaker species' quota is filled faster, so stronger species are
under-exploited in order to reduce the fishing pressure on the weakest,
or pinch-point, species (Figure 3) (Hilborn et al. (2004)). An example
of a pinch-point species is Boccacio rockfish (\emph{S. paucispinis}) in
the British Columbia groundfish fishery, which are difficult to avoid
when targeting lingcod (\emph{Ophiodon elongatus}). Bocaccio rockfish
are listed as Endangered by COSEWIC\footnote{Committee on the Status of
  Endangered Wildlife in Canada.} and have a very low annual quota of
around 110 metric tonnes (mt), while lingcod are highly productive with
annual quota of around 3600mt. Avoidance of Bocaccio by harvesters led
to less than 33\% of Bocaccio quota to be utilised between 2006 and 2014
(Figure 4). Technical interactions between Bocaccio and lingcod means
that this avoidance behaviour resulted in around 25\% of lingcod quota
being utilised in that same time period (Figure 4). This
underutilisation translates into a reduction of around \textbf{DOLLAR
AMOUNT}\footnote{February, 2016 prices \textbf{HEY DUMMY, ADJUST THIS
  FOR NPV}} gross revenue to the BCIGF between 2006 and 2014.

\subsection{Assess and Avoid}\label{assess-and-avoid}

Profitability constraints caused by technical interactions may be
alleviated by conducting stock assessments of data-limited species and
avoiding pinch-point species. Species that lack up-to-date assessments
often have their quota set to a low level for conservation reasons,
creating artificial pinch-points. After assessment the quota of a
data-limited species can be scaled to a better estimate of stock-status
(Food and Agriculture Organization of the United Nations (1995)), which
could have 2 effects. Either the status is such that the pinch-point
created by the data-limitation can be removed, or the status requires
the pinch-point to remain. In the case where assessments show that the
pinch-point cannot be removed then an avoidance strategy is required.

One option for overcoming data limitations to assessments is by
explicitly acknowledging technical interactions in assessment models
(Mueter \& Megrey (2006); A. E. Punt, Smith, \& Smith (2011); Zhou et
al. (2010)). Technical interactions can be acknowledged by aggregating
multiple species into the same assessment complex or assemblage based on
co-occurence in fishing events, similar to Pacific salmon runs (Beverton
et al. (1984); Walters \& Martell (2004)). Statistical benefits of
aggregation may allow previously unassessed species to be assessed, and
increase the profitability of the fishery by relieving constraints and
enabling eco-certification. While more complicated than the single
specie paradigm, the benefit of assessing previously unassessed species
may outweigh the costs.

\textbf{Based on a re-ordering of the diagram} Figure 6 shows three
possible models of fishery operation and management. Models (a) and (b)
are the current options for assessment in multispecies fisheries. Model
(a) is the status quo approach of single species stock assessment, where
every stock is treated as a separate population (Hilborn \& Walters
(1992)). Model (b) is the total aggregation approach used for Pacific
salmon (English et al. (2011)), where several species or stocks have
their data combined and are then assessed and managed as a single unit
(Sugihara et al. (1984); Gulland \& Garcia (1984); Gaichas et al.
(2012)).

The total aggregation approach used by Pacific salmon may not be
suitable for assessing assemblages of multiple species with distinct
life histories and reproductive strategies. Model (c) in Figure 5
addresses this by keeping the data separate as in model (a), but
performs assessments for groups of stocks using statistical models that
link the data during estimation (Zhou et al. (2010); A. E. Punt et al.
(2011); Mueter \& Megrey (2006)).

In Chapters 1, 2 and 3 I conduct a simulation study of a hierarchical
stock assessment model to share data between species as in model(c) of
Figure 5 (Jiao, Hayes, \& Cortés (2009); Zhou et al. (2010); A. E. Punt
et al. (2011)). The statistical model assumes a hierarchical structure
of multispecies fisheries as shown in Figure 7, allowing for an an
intermediate level of aggregation between models (a) and (b) of Figure
6. Shared parameters in the hierarchical assessment model allow for some
of the benefits of aggregation for data-limited stocks, but the
separated data streams allows for species specific estimates of
abundance and productivity (Jiao et al. (2009)).

The focus of Chapter 1 is to create a simulation-estimation procedure to
study hierarchical assessment models for multi-species assemblages with
no sub-stock structure. Data generated by a process error population
dynamics model and observation model are provided to hierarchical
estimators (Zhou et al. (2010); A. E. Punt et al. (2011)). The
statistical performance of the estimators is then quantified by
comparing the true values of parameters to estimated values.

In Chapter 2, the simulation-estimation procedure incorporates a
sub-stock structure for each species. Including multiple sub-stocks that
correspond to possible multi-stock structure increases the resolution of
the data and allows for deviations from average life history parameter
values within each species (Su, Peterman, \& Haeseker (2004)). Bias and
precision are estimated and compared to the results of Chapter 1, to
analyse the benefits and costs of including increased structure in the
model.

In Chapter 3, a closed loop feedback simulator is used to evaluate
management procedures using hierarchical models (Figure 8). This
involves creating an operating model that simulates population dynamics
of multiple interacting fish species, effort dynamics of multiple
fishing fleets with different gear types exploiting those populations
and uncertain observations made by scientific surveys (Hilborn \&
Walters (1987); Walters \& Bonfil (1999); M. L. Jones et al. (2009);
Clark (2010)). Uncertain data provided by the operating model reacts
with the management procedure to produce complex emergent properties and
closed loop simulation offers a low-stakes option for analysing those
properties and the associated risks.

In Chapter 4, I investigate a data-based approach to avoiding non-target
species and estimate its economic value. Reliable, spatially explicit
commercial data is becoming more abundant with increasing observer
coverage in modern fisheries. Concurrent with this, machine learning
methods are emerging that allow for analysis of data that isn't
collected under strict experimental designs (T. Hastie et al. (2009)),
such as commercial fishing data.

\subsection{Study System}\label{study-system}

The British Columbia Integrated Groundfish Fishery (BCIGF) (Fisheries
and Oceans, Canada (2015)) is a group of 7 fisheries that spatially and
temporally overlap on the BC coast. The overlapping fisheries are
managed by one integrated individual transferrable quota system,
allowing temporary and permanent transfers of quota allocations between
licenses in different fleets. All catch and discards are deducted from
quota allocations, and are therefore monitored on 100\% of vessels by at
sea observer or electronic monitoring systems. Skippers who exceed their
quota share must either obtain more from other harvesters, or stop
fishing for the season.

Integrated management of the BCIGF creates pinch-points on quota
utilisation, caused by technical interactions between directed species
and data limited non-target species. Many species lack up-to-date
assessments (Driscoll (2014)) creating artificial pinch-points that
could be alleviated by assessing and avoiding those species.

In Chapters 1, 2 and 3 the simulation study uses a multispecies complex
composed of all flatfish except halibut in the BCIGF as the biological
component of the operating model. The complex is made up of
\textbf{D}over sole (\emph{Microstomus pacificus}), \textbf{E}nglish
sole (\emph{Parophrys vetulus}), \textbf{R}ock sole (\emph{Lepidopsetta
bilineata}), \textbf{P}etrale sole and \textbf{A}rrowtooth flounder
(\emph{Atheresthes stomias}) (Fisheries and Oceans, Canada (2015)), and
called the \textbf{DERPA} complex for brevity. All members of DERPA are
from the family \emph{Pleuronectidae} of right-eyed flounders, making
DERPA suitable for a hierarchical approach due to similar but distinct
life histories. The amount of data available for different DERPA species
varies, with Rock sole being subject to regular assessments, and Petrale
sole having no up to date assessments (Driscoll (2014)). Halibut are
excluded as they are managed by a separate trans-boundary authority.

In Chapter 4, I use machine learning methods to forecast the presence of
sub-legal sized sablefish in fishing events. Sablefish are at historic
low abundances and are subject to a rebuilding strategy (\textbf{REFS}).
Reducing discard induced mortality of juvenile sablefish may be an
alternative to quota reduction for increasing spawning stock biomass
(\textbf{STOCK ASSESSMENT REFERENCE}). Discarding of legal-sized
sablefish (\textgreater{}55cm, good condition) is economically
disincentivised by a quota deduction adjusted for discard induced
mortality, but no such incentive or mortality rate exists for
unmarketable sablefish (\textless{}55cm, poor condition). This incentive
structure is evident in the distribution of sablefish discarding, with
\textbf{CONCRETE NUMBERS}\% of sablefish discards made up by sub-legal
sized fish.

\section{Chapter 1: Estimating Coastwide Abundance and Productivity in a
Multispecies Groundfish Fishery via a Hierarchical Stock Assessment
Model}\label{chapter-1-estimating-coastwide-abundance-and-productivity-in-a-multispecies-groundfish-fishery-via-a-hierarchical-stock-assessment-model}

\subsection{Background}\label{background-1}

Quantitative stock assessment models incorporate population dynamics
processes (Figure 1.1), observational data (Figure 1.2) and a
statistical model (Figure 1.3) (Hilborn \& Walters (1992)). Model inputs
are candidate parameter values that are confronted by data in the
statistical model to produce posterior density or likelihood values as
outputs. Statistical model output is optimised or integrated over the
input parameters to extend inferences about stock productivity and
status in the form of distributional estimates.

Hierarchical statistical models are becoming increasingly popular for
analysing complex fisheries data. In Pacific salmon stock and
recruitment analyses, both Bayesian and frequentist (mixed effects)
hierarchical models are used in meta-analyses of multistock populations
(Su et al. (2004); Malick, Cox, Mueter, Peterman, \& Bradford (2015)).
More related to this thesis, stock assessment models that use
hierarchical statistical models are sometimes used to assess
multispecies complexes where data limitations are an issue for single
species management, such as technical interactions between data-limited
species (A. E. Punt et al. (2011)) or difficulties in species
identification (Jiao et al. (2009)).

In this chapter, I use a simulation-estimation procedure to study
hierarchical Bayesian (Zhou et al. (2010)) and frequentist (A. E. Punt
et al. (2011)) state space multispecies assessment models. The
multispecies models are used to simultaneously assess a simulation of
the DERPA complex of flatfish. In a comparison between single species
and hierarchical models applied multispecies groups including
data-limited species, it has been shown that the hierarchical models
induce a change in parameter estimates for data-limited species (A. E.
Punt et al. (2011); Kell \& De Bruyn (2012)). However, it is unknown if
that change is an increase or decrease in bias.

\textbf{QUESTION:} What is the bias in estimates for abundance and
productivity produced by hierarchical multispecies models?

Simulated scientific and commercial data are used to test hierarchical
assessment models. True parameter values used for simulation can be
compared to estimated parameters in Monte-Carlo trials to understand
bias and precision of both estimators. Estimators are then tested across
a range of scenarios representing implications of technical interactions
between species, and contrasts in data availability.

\subsection{Methods}\label{methods}

Each species in the DERPA complex simulated independently using the
model defined in Table 2. Population dynamics are simulated by a simple
biomass dynamics process error-model (Figure 1.1, Eqs T2.2, T2.4),
fishery dependent catch is generated using fishing mortality as an input
(Eq T2.3) and fishery independent observations of catch per unit effort
(CPUE) are generated by the observational model (Figure 1.2, Eq T2.5).

Multispecies data produced by the simulation model are supplied to both
a Bayesian and frequentist version of a hierarchical state-space
assessment model (Figure 7). Both assessment models are specified in the
same way, shown in Table 3. The difference between the models is in how
the inferences are extended. For the Bayesian state space model the
posterior density (Eq. T3.8) is integrated over all input parameters
producing marginal distributions for each parameter (Gelman, Carlin,
Stern, \& Rubin (2014)). For the frequentist state space model, also
known as a random effects model, the posterior density is integrated
over the priors to produce a marginal ``true'' likelihood, which is then
maximised as in traditional likelihood methods (de Valpine \& Hastings
(2002)).

Both models require a numerical integration method to produce marginal
distributions or likelihoods (de Valpine \& Hastings (2002); Gelman et
al. (2014), Maunder, Deriso, \& Hanson (2015)). Integration generally
requires numerical methods like Markov-Chain Monte Carlo (MCMC)
algorithms for distribution sampling of complex non-linear, non-Gaussian
statistical models. To this end, the Bayesian model is coded using the
Automatic Differentiation Model Builder (ADMB) suite (Fournier et al.
(2012)) and the random effects model using Template Model Builder
(Kristensen, Nielsen, Berg, Skaug, \& Bell (2015)). Both software
packages provide fast numerical integration to produce marginal
distributions, with TMB being developed specifically for random effects
models.

Model testing proceeds through four experimental scenarios that modify
simulation input parameters representing multispecies interactions and
data limitations. Parameter estimates from each trial are then compared
to their true values generated by the simulator to estimate bias and
precision of the models in each scenario.

The first two scenarios investigate model assumptions about process
error deviations \(\epsilon_{s,t}\) and species catchability
coefficients \(q_s\). Both parameters are representive of interactions
between species in the complex. For example, species that share the same
habitat will encounter the same environmental variation, reflected in
process error deviations (A. E. Punt et al. (2011)). Similarly, species
that are fished by the same gear may have similar interactions with
fishing gear leading to correlations in catchability.

Shared priors are defined for process error deviations (T3.7) and
catchability parameters (T3.8). Bias and precision are measured for a
range of fixed values of the prior variance (eg
\(\sigma^2 \in (0,\infty)\)) (Gelman et al. (2014), Ch 5.5).

The remaining two scenarios test contrasts in observation error variance
\(\tau_s^2\) between species and fishery development histories
\(F_{s,t}\), representing data or information available to fishery
managers. Observation error is a direct measurment of the quality of
data obtained by scientific surveys, so contrasts in observation error
variance \(\tau_s^2\) simulate differing levels of data availability
between species in an assemblage. Fishery development histories are also
a source of information, due to the way a fish population will respond
to differing levels of exploitation (Hilborn \& Walters (1992), Ch 2).
\textbf{MAKE FIGURE of Ft trajectories}

\subsection{Expected Results}\label{expected-results}

I expect this chapter to result in a working knowledge of how
hierarchical stock assessment models change the estimates of abundance
and productivity when applied to multispecies assemblages. Estimates of
model bias and precision as functions of correlation strengths,
observation error variance and historical fishing are produced. Results
are to be published in a paper about the statistical properties of 2
hierarchical multispecies assessment models.

Assumptions about the strength of correlations in shared parameters are
likely to introduce bias through shrinkage towards a mean (Mueter,
Peterman, \& Pyper (2002)). The extent of the shrinkage introduced can
be understood by producing bias and precision estimates under a range of
fixed values of shared prior variance.

The extent to which limitations on data and species specific information
can be overcome (A. E. Punt et al. (2011)), if at all, can be quantified
through bias and precision estimates resulting from scenarios
contrasting data-availability and fishing histories. This is especially
helpful for fisheries in which there are limited historical fishing and
scientific data available, or limited resources for improving existing
scientific surveys.

\section{Chapter 2: Adding Multistock Structure to Multispecies
Hierarchical Stock Assessment
Models}\label{chapter-2-adding-multistock-structure-to-multispecies-hierarchical-stock-assessment-models}

\subsection{Background}\label{background-2}

A high degree of spatial variation in genetics, morphology, life-history
and behaviour is apparent in many exploited fish populations (Hilborn,
Quinn, Schindler, \& Rogers (2003); Schindler et al. (2010)). Management
of exploited fishes without acknowledgement of this variation risks
eroding biodiversity and increasing species vulnerability to
environmental variation (Hilborn et al. (2003); Cope \& Punt (2011);
Ashleen J Benson et al. (2015)).

Aggregation of sub-stocks into a single management unit over large
spatial scales relies on assumed rescue effects that do not exist in
general multistock fisheries. The assumption is that despite spatial
disaggregation of the stock, sub-stocks are connected by migration
creating a rescue effect (Dulvy, Sadovy, \& Reynolds (2003)). Rescue
effects are then believed to reduce the risks of managing spatially
complex species in a single aggregate (Cope \& Punt (2011)). However,
this rescue effect is highly dependent on dispersal and recruitment
patterns in the meta-population and individual natural mortality rates
of sub-stocks (Ashleen Julia Benson (2011)).

When stock structure is easily identified, as with Pacific salmon, there
are advantages to managing a species at the level of individual stocks.
For example, by estimating productivity levels for 43 individual stocks
of Pink salmon the effects of local variation in sea surface temperature
could be discovered (Su et al. (2004)). Furthermore, estimating
individual productivity levels within a management complex reduces the
risk of overfishing weak stocks due to an averaging effect (Figure 2).

Managing multistocks also has its challenges. When the exact nature and
connectedness of the spatial stock structure is unknown, it is unclear
whether or not aggregation is the more precautionary management approach
(Ashleen J Benson et al. (2015)). Furthermore, for a data-limited
species further disaggregation of the data will only raise further
barriers to stock assessment by reducing the amount of data available
for each sub-stock.

A hierarchical stock assessment model may overcome data limitations from
disaggregation when managing for multiple stocks in a multispecies
fishery (A. E. Punt et al. (2011)). Life histories within species are
likely to be similar, allowing for prior distributions on life history
parameters that are shared between stocks. Similarly, sub-stocks of
multiple species share habitat and experience the same environmental
variation, allowing for a local spatial effect on process error
(Kallianiotis, Vidoris, \& Sylaios (2004)).

\textbf{QUESTION:} How do estimates of abundance and productivity in a
multistock, multispecies hierarchical model compare to those of a
coastwide multispecies hierarchical model?

The DERPA complex exhibits evidence of sub-stock structure. For example,
the species population of English sole on the British Columbia coast is
managed as two segregated major stocks with limited migration (Hart,
Clemens, \& others (1973)). Simulated data from a multi-stock model of
the DERPA complex is provided to both a coastwide and multistock
hierarchical multispecies model. Both models produce parameter
estimates, and bias and precision are compared.

\subsection{Methods}\label{methods-1}

The DERPA complex is simulated as individual stocks \(j\) of each
species \(s\) (Table 5). Migration from stock \(i\) to stock \(j\)
within species \(s\) is possible with net migration rate
\(\phi_{s,i,j}\), making stock dynamics interdependent (T5.4).

The estimation procedure includes a layer of hierarchical structure to
include multiple substocks for each species (Figure 8(b)). The multiple
stocks within a species share prior distributions on life history
parameters at the species level. The hyperparameters of shared priors at
the species level then share hyperpriors with other species at the
assemblage level.

Six experimental scenarios evaluated by estimating bias and precision
extend the four outlined in Chapter 1. The original four will be
extended to account for the increased depth in the assemblage structure.
The first additional scenario models increased data-limitation
introduced by disaggregating an already data-limited species into
multiple stocks. Disaggregation could lead to increased observation
error variance or entirely missing observations for some stocks. The
final additional experiment introduces spatial covariation in the state
dynamics simulation for each stock (\textbf{FIGREF??}) and tests the
benefit of including spatial variation in the estimator.

\subsection{Expected Results}\label{expected-results-1}

I expect this chapter to deepen understanding of hierarchical estimators
and their application in a multistock context. Adding stock structure
involves increased model complexity and reduced data availability due to
disaggregation, introducing a tradeoff. This tradeoff is then evaluated
by varying data availability and model complexity and examining how
model bias and precision change. A publication detailing the tradeoffs
between bias and precision under different model structures is expected
to result from this analysis.

\section{Chapter 3: Management Performance of Hierarchical Multispecies
Assessment
Models}\label{chapter-3-management-performance-of-hierarchical-multispecies-assessment-models}

\subsection{Background}\label{background-3}

The fisheries management procedure extends beyond the stock assessment
model (Figure 1). Stock assessment output (Figure 1.3) informs a
decision rule (Figure 1.4) that determines the amount of fishing effort
expended to collect the harvest quota (Figure 1.5). This effort
dynamically impacts fish populations and their habitat (Figure 1.1),
providing new data (Figure 1.2) that is used for assessment.

An important test for an assessment model is how it performs as part of
a management procedure. Management procedures include harvest
strategies, which are input or output controls on the fishery (Hilborn
\& Walters (1992), Ch. 15) and decision rules that scale controls to
stock status. Management procedures made up of decision rules, harvest
strategies and assessment models represent the full management cycle of
a fishery.

In this chapter, I use closed loop simulation to test management
procedures based on a hierarchical multispecies stock assessment model.
Closed loop simulation modeling explicitly quantifies feedback in a
dynamic system (de la Mare (1998); Sainsbury, Punt, \& Smith (2000)). In
a fisheries management context, the closed loop includes the management
procedure, fish stocks and commercial and scientific data in a feedback
loop (Figure 9). The fishery, population dynamics and scientific survey
are part an operating model (M. L. Jones et al. (2009)) that provide
data to the assessment model and harvest control rule as part of a
management procedure. Management procedure evaluation then proceeds by
experimentally adjusting model parameters and observing the emergent
behaviour. In this way, potential risks of management can be quantified
under a given set of assumptions.

Simulating a multistock, multispecies fishery requires a more
sophisticated simulator than in Chapters 1 and 2. For example,
harvesters expend fishing effort based on expected costs and benefits of
fishing, including expected catch composition and personal risk.

\textbf{QUESTION:} How can realistic targeting behaviour be included in
the operating model?

Targeting behaviour comes down to when and where harvesters expend their
effort. That is, targeting behaviour can be simulated by including
fishing effort dynamics for multiple fishing fleets (gear types) in the
operating model (Hilborn \& Walters (1987)). These dynamics are based on
estimates of fishery dependent catchability \(q_{f,s,t}\) (Table 7) can
be empirically drawn from commercial data or simulated parametrically.

Once realistic effort dynamics are included in the operating model,
inherent risks of assessment model assumptions can be assessed across
multiple experimental scenarios. Experiments include contrast in
data-quality, taking into account spatial covariation due to
environmental forcing (Dichmont, Deng, Punt, Venables, \& Haddon (2006))
.

\textbf{QUESTION:} How do multispecies hierarchical assessment models
perform when managing multispecies assemblages containing data limited
stocks, with performance measured by probability of overfishing,
underfishing and variation in annual catch?

I answer this question by running experimental scenarios that measure
the relationship between management risks and data quality. For example,
risk countour plots (\textbf{FIGURE}) could help inform policy for the
application of hierarchical assessment models or communication with
stakeholders when providing harvest advice.

\subsection{Methods}\label{methods-2}

The closed loop simulation extends the simulation model of Chapter 2
into an operating model including effort dynamics (Tables 7, 8 \textbf{8
not done yet}). At each time step \(t\), the current state of each fish
population is estimated by the assessment model. Asessment models then
forecast abundance at time \(t+1\), which is passed through a harvest
control rule (HCR) to generate a total allowable catch (TAC) or each
species. The TAC for each species is then supplied to the operating
model, which distributes fishing effort (Hilborn \& Walters (1987))
across space in order to maximise some objective, such as fishing
profit, subject to the constraints of the TAC.

Fishing effort dynamics are simulated by through a 2 stage procedure at
each time step (Hilborn \& Walters (1987)). First, some test fishing is
conducted by expending a unit of effort in each fleet and fishing
location (statistical area). Then, based on fleet objectives and
observed fishery dependent catch rates, the operating model will
distribute effort to maximise the objective subject to the TAC for every
species. It is possible that these dynamics will lead to underutilised
quota for some species, effectively simulating pinch-points.

Management procedures featuring assessment models from Chapters 1 and 2
areused in experimental scenarios. Experiments test a range of
observation error variances, process error variances, fishery
development history and correlations in catchability \(q_{s,j,t}\). Each
simulation measures quota utilisation, species depletion, probability of
exceeding optimal instantaneous fishing mortality and annual average
variation. Simulation output is then used to compare between scenarios
and management procedures, quantifying performance and risks of each
procedure.

\subsection{Expected Results}\label{expected-results-2}

For each experiment a plot of overfishing risk as a function of contrast
like Figure (\textbf{FIGURE}) will be produced. This plot will show
contours of probability that the actual fishing mortality rate \(F_t\)
is greater than the fishing mortality rate giving MSY \(F_{MSY}\). The
\(y\)-axis will show target fishing mortality \(F_t\) or effort \(E_t\),
and the \(x\)-axis will show the parameter driving the contrast.

\begin{itemize}
\tightlist
\item
  Contour plot
\item
  Operating model
\item
  Paper
\end{itemize}

\section{Chapter 4: Avoiding non-target
species.}\label{chapter-4-avoiding-non-target-species.}

\subsection{Introduction}\label{introduction-1}

Quota on directed species in the BCIGF is constrained by restrictive
quotas on pinch-point species and size limits on directed species, both
of which are caught incidentally during directed fishing. For example,
an average of 160 tonnes per year of Sablefish below 55cm in length were
discarded due to size regulations by trap and trawl fishing vessels
between 2007 and 2015 in the BCIGF \textbf{SOURCE: GFFOS}.

There is general agreement in the literature that incidental catch and
discarding should be reduced as much as possible or practical (Saila \&
Jones (1983); Crowder \& Murawski (1998); Safina \& Lewison (2008); Pelc
et al. (2015)). Mortality of immature individuals caused by unregulated
or regulated (i.e.~size, quota, trip limits) discarding contributes to
both recruitment and growth overfishing of fish stocks (Crowder \&
Murawski (1998)). Furthermore, bycatch has an impact on the ecosystem
containing the target resource, including all non-resource species and
habitats that interact with the fishing gear (Safina \& Lewison (2008)).

In this chapter I test the feasibility of using a model-based approach
to predicting fishing events that encounter and discard juvenile
sablefish by analysing commercial fishing data for the purposes of
avoiding regulatory discarding. Data generated by commercial fishing is
not randomly sampled, so traditional statistical models that rely on the
central limit theorem are unsuitable. Instead, machine learning models
are used to sidestep statistical assumptions and search for correlations
in the data (C. M. Bishop (2006); T. Hastie et al. (2009)).

Three machine learning models are trained and optimised on a subset of
commercial data from the BCIGF to classify presence and absence of
juvenile sablefish for a given fishing event. Event predictions from all
three models are then combined into an ensemble classifier (Rokach
(2010)). Ensemble classifiers use weighted model averaging techniques to
overcome potential overfitting to the training data. The remaining data
is then used to test the performance of the ensemble classifier using
multiple metrics (Freeman \& Moisen (2008))

The feasibility of a tool to avoid regulatory discarding requires an
economic benefit to harvesters. Because juvenile Sablefish are discarded
under size regulations, no discard induced mortality is deducted from
harvester quota (Fisheries and Oceans, Canada (2015)). No reduction in
quota implies a lack of economic incentive for harvesters to avoid
conditions that lead to the catch of juvenile fish.

\textbf{QUESTION:} What is the benefit of using a machine learning
approach for predicting the presence and absence of juvenile sablefish
in commercial fishing events, compared to the status quo?

Economic benefit of the ensemble classifier is measured by estimating
the value of information provided by the classifier (Mäntyniemi, Kuikka,
Rahikainen, Kell, \& Kaitala (2009)). Classifier performance is combined
with empirical estimates of the probability of encounter in a decision
analysis, with utility provided by a dollar value based on the costs and
benefits of successful and unsuccessful avoidance.

\subsection{Methods}\label{methods-3}

Predictive capacity of an ensemble machine learning classifier to detect
juvenile Sablefish is tested on 4 sets of commercial fishing data from
the BCIGF. The BCIGF data is contained in the Groundfish Fishery
Operating System (GFFOS) data base that contains spatially and
temporally explicit data for every fishing event in the BCIGF since
2005. The 4 data sets are split by gear type, with data sets containing
events using (i) trawl only, (ii) longline trap only, (iii) longline
hook only and (iv) all gear types.

For each data set, an ensemble classifier is built from Random Forest
(Breiman (2001)), Naive Bayes (ref) and Artificial Neural Network (ref)
component classifiers. Component classifier configurations are chosen
based on average performance over Monte-Carlo trials of a validation
procedure (T. Hastie et al. (2009), Ch 7.2). Performance of classifiers
is measured using multiple metrics including percentage correctly
classified, area under receiver operating characteristic curves,
precision and recall (Freeman \& Moisen (2008)). Component classifiers
that perform the best are then combined into an ensemble using multiple
configurations, ranging from simple model stacking to Bayesian Model
combination (Rokach (2010)). Ensemble classifiers are then tested on a
reserved portion of the data to estimate the classification error rate
of the ensemble.

A formal decision analysis is performed to estimate the value of
information provided by using the classifier with the lowest error rate
on each data set (Mäntyniemi et al. (2009); Peterman \& Anderson (1999);
Pestes, Peterman, Bradford, \& Wood (2008)). Classifiers are included in
the analysis as a form of expert opinion, adjusting the probability of
encountering juvenile Sablefish given a message from the classifier. The
value of information is then the difference in the expected utility of
fishing with the classifier's help and the expected utility of fishing
without it. The utility is dependent on the costs setting gear and
sorting discards from the catch, as well as the value of landed catch.

\subsection{Expected Results}\label{expected-results-3}

\begin{itemize}
\tightlist
\item
  A conclusion on the feasibility of avoidance based on commercial data
  retained by monitoring.
\item
  Maps of encounter probability for ease of communication to harvesters
\item
  The basis of a fleet communication system for avoidance of non-target
  species and possible DOM
\item
  Estimates of the economic benefit of a tool for avoiding non-target
  species.
\end{itemize}

\section{Conclusions}\label{conclusions}

This thesis will undertake a study of stock assessment methods for
assemblages of multiple interacting species. The focus is on the Robin
Hood (RH) method, which contains a hierarchical statistical model
linking the data of multiple species. The RH method could prove valuable
for data-variable fisheries that are unable to produce stock assessments
for all species that they contain. Those species that were previously
unable to be assessed could potentially fall inside the stable risk
region of the RH method allowing scientifically defensible TACs to be
set. This would help some fisheries obtain eco-certification (MSC,
Seafood Watch, seaChoice), helping to improve biological sustainability
of market share.

Further to contributions to the field of fisheries science, this thesis
also satisfies the interdisciplinary spirit of REM in two chapters.
First, assessing multiple species simultaneously using the RH approach
brings possible advantages, such as an efficient allocation of survey
resources. Such advantages are to be tested in Chapter 2 through
management strategy evaluation, synthesising ecological drivers of
population dynamics and economic drivers of fishing effort. The other
methods - single species (SS) and total aggregation (TA) - are analysed
as alternatives to the RH method in a risk assessment in Chapter 3. The
risk assessment is intended to recommend data-requirements and harvest
control rules for stable social and biological risk, synthesising
economics, ecology and policy.

\subsection{Resource allocation improvement (taken from
intro)}\label{resource-allocation-improvement-taken-from-intro}

Efficient allocation of resources is therefore an important part of the
sustainable management of multispecies fisheries. This para needs refs,
better ideas and structure. Does it even go here? Maybe combine it with
the following paragraph on multispecies assessments. Also, include
information here about ``good enough'' mechanisms for resource
distribution • SARA, PA state that at a given level of biological risk,
a recovery process has to be implemented. • Are expensive, model based
assessments required for every species? • What constitutes ``good
enough'' management? • There are two parts to resource allocation 1.
Identification of need - pinch-point species are an externality on
directed species 2. Mechanisms for resource distribution - good enough
MPs: Swept-area, total Agg, RH\ldots{} • A paralysis of choice is common
in situations like this.

I propose a practical, stake-holder driven management approach that
seeks to to satisfy conservation concerns while increasing economic
welfare in multispecies fisheries. Assessment techniques should be
\emph{practical} in that they achieve conservation and management goals
efficiently, such as reserving the most technologically advanced and
resource intensive management procedures for those stocks that
absolutely need it. Management decisions should be \emph{stake-holder
driven}, where priority is given to actions that increase the economic
welfare of the fishery. This will allow for positive economic feedback
and increase the pool of resources available for future management of
the fishery. The goal of a practical, stake-holder driven management
approach is to simultaneously improve the sustainability
(intergenerational equity) and profitability of multispecies fisheries.

\section{Biblography}\label{biblography}

\hypertarget{refs}{}
\hypertarget{ref-baudron2015adverse}{}
Baudron, A. R., \& Fernandes, P. G. (2015). Adverse consequences of
stock recovery: European hake, a new ``choke'' species under a discard
ban? \emph{Fish and Fisheries}, \emph{16}(4), 563--575.

\hypertarget{ref-begg1999stock}{}
Begg, G. A., Friedland, K. D., \& Pearce, J. B. (1999). Stock
identification and its role in stock assessment and fisheries
management: An overview. \emph{Fisheries Research}, \emph{43}(1), 1--8.

\hypertarget{ref-benson2011evaluation}{}
Benson, A. J. (2011). \emph{An evaluation of the opportunities and
impediments in managing quota fisheries for biodiversity} (PhD thesis).
Environment: School of Resource; Environmental Management.

\hypertarget{ref-benson2015evaluating}{}
Benson, A. J., Cox, S. P., \& Cleary, J. S. (2015). Evaluating the
conservation risks of aggregate harvest management in a
spatially-structured herring fishery. \emph{Fisheries Research},
\emph{167}, 101--113.

\hypertarget{ref-beverton1984dynamics}{}
Beverton, R., Cooke, J., Policansky, D., Csirke, J., Roughgarden, J.,
Doyle, R., \ldots{} others. (1984). Dynamics of single species. In
\emph{Exploitation of marine communities} (pp. 13--58). Springer.

\hypertarget{ref-bishop2006pattern}{}
Bishop, C. M. (2006). \emph{Pattern recognition and machine learning}
(Vol. 4). springer New York.

\hypertarget{ref-breiman2001random}{}
Breiman, L. (2001). Random forests. \emph{Machine Learning},
\emph{45}(1), 5--32.

\hypertarget{ref-clark2010mathematical}{}
Clark, C. W. (2010). \emph{Mathematical bioeconomics: The mathematics of
conservation} (Vol. 91). John Wiley \&amp; Sons.

\hypertarget{ref-cope2011reconciling}{}
Cope, J. M., \& Punt, A. E. (2011). Reconciling stock assessment and
management scales under conditions of spatially varying catch histories.
\emph{Fisheries Research}, \emph{107}(1), 22--38.

\hypertarget{ref-crowder1998fisheries}{}
Crowder, L. B., \& Murawski, S. A. (1998). Fisheries bycatch:
Implications for management. \emph{Fisheries}, \emph{23}(6), 8--17.

\hypertarget{ref-de-la-Mare1998Tidier-fisherie}{}
de la Mare, W. K. (1998). Tidier fisheries management requires a new mOP
(management-oriented paradigm). \emph{Reviews in Fish Biology and
Fisheries}, \emph{8}(3), 349--356.

\hypertarget{ref-de2002fitting}{}
de Valpine, P., \& Hastings, A. (2002). Fitting population models
incorporating process noise and observation error. \emph{Ecological
Monographs}, \emph{72}(1), 57--76.

\hypertarget{ref-DFO1999Fraser-River-Ch}{}
DFO. (1999). \emph{Fraser river chinook salmon} (No. D6-11). DFO Science
Stock Status Report.

\hypertarget{ref-dichmont2006management2}{}
Dichmont, C. M., Deng, A. R., Punt, A. E., Venables, W., \& Haddon, M.
(2006). Management strategies for short lived species: The case of
australia's northern prawn fishery: 2. choosing appropriate management
strategies using input controls. \emph{Fisheries Research},
\emph{82}(1), 221--234.

\hypertarget{ref-Driscoll2014Groundfish-comp}{}
Driscoll, J. (2014). \emph{Groundfish complex, british columbia}.
Seafood Watch, Monterey Bay Aquarium.

\hypertarget{ref-dulvy2003extinction}{}
Dulvy, N. K., Sadovy, Y., \& Reynolds, J. D. (2003). Extinction
vulnerability in marine populations. \emph{Fish and Fisheries},
\emph{4}(1), 25--64.

\hypertarget{ref-english2011fraser}{}
English, K. K., Edgell, T. C., Bocking, R. C., Link, M. R., \& Raborn,
S. W. (2011). Fraser river sockeye fisheries and fisheries management
and comparison with bristol bay sockeye fisheries. \emph{The Cohen
Commission of Inquiry into the Decline of Sockeye Salmon in the Fraser
River}.

\hypertarget{ref-PRIFMP2015}{}
Fisheries and Oceans, Canada. (2015). Pacific Region Integrated
Fisheries Management Plan: Groundfish.

\hypertarget{ref-FAO1995}{}
Food and Agriculture Organization of the United Nations. (1995). Code of
conduct for responsible fisheries.

\hypertarget{ref-fournier2012ad}{}
Fournier, D. A., Skaug, H. J., Ancheta, J., Ianelli, J., Magnusson, A.,
Maunder, M. N., \ldots{} Sibert, J. (2012). AD model builder: Using
automatic differentiation for statistical inference of highly
parameterized complex nonlinear models. \emph{Optimization Methods and
Software}, \emph{27}(2), 233--249.

\hypertarget{ref-freeman2008comparison}{}
Freeman, E. A., \& Moisen, G. G. (2008). A comparison of the performance
of threshold criteria for binary classification in terms of predicted
prevalence and kappa. \emph{Ecological Modelling}, \emph{217}(1),
48--58.

\hypertarget{ref-gaichas2012assembly}{}
Gaichas, S., Gamble, R., Fogarty, M., Benoît, H., Essington, T., Fu, C.,
\ldots{} Link, J. (2012). Assembly rules for aggregate-species
production models: Simulations in support of management strategy
evaluation. \emph{Marine Ecology Progress Series}, \emph{459}, 275--292.

\hypertarget{ref-gelman2014bayesian}{}
Gelman, A., Carlin, J. B., Stern, H. S., \& Rubin, D. B. (2014).
\emph{Bayesian data analysis} (Vol. 2). Taylor \&amp; Francis.

\hypertarget{ref-gulland1984observed}{}
Gulland, J., \& Garcia, S. (1984). Observed patterns in multispecies
fisheries. In \emph{Exploitation of marine communities} (pp. 155--190).
Springer.

\hypertarget{ref-hart1973pacific}{}
Hart, J. L., Clemens, W. A., \& others. (1973). \emph{Pacific fishes of
canada}. Fisheries Research Board of Canada.

\hypertarget{ref-hastie2009elements}{}
Hastie, T., Tibshirani, R., Friedman, J., Hastie, T., Friedman, J., \&
Tibshirani, R. (2009). \emph{The elements of statistical learning} (Vol.
2). Springer.

\hypertarget{ref-hilborn1987general}{}
Hilborn, R., \& Walters, C. J. (1987). A general model for simulation of
stock and fleet dynamics in spatially heterogeneous fisheries.
\emph{Canadian Journal of Fisheries and Aquatic Sciences}, \emph{44}(7),
1366--1369.

\hypertarget{ref-hilborn1992quantitative}{}
Hilborn, R., \& Walters, C. J. (1992). \emph{Quantitative fisheries
stock assessment: Choice, dynamics and uncertainty/Book and disk}.
Springer Science \&amp; Business Media.

\hypertarget{ref-hilborn2004beyond}{}
Hilborn, R., Punt, A. E., \& Orensanz, J. (2004). Beyond band-aids in
fisheries management: Fixing world fisheries. \emph{Bulletin of Marine
Science}, \emph{74}(3), 493--507.

\hypertarget{ref-hilborn2003biocomplexity}{}
Hilborn, R., Quinn, T. P., Schindler, D. E., \& Rogers, D. E. (2003).
Biocomplexity and fisheries sustainability. \emph{Proceedings of the
National Academy of Sciences}, \emph{100}(11), 6564--6568.

\hypertarget{ref-hutchings2012canada}{}
Hutchings, J. A., Côté, I. M., Dodson, J. J., Fleming, I. A., Jennings,
S., Mantua, N. J., \ldots{} VanderZwaag, D. L. (2012). Is canada
fulfilling its obligations to sustain marine biodiversity? A summary
review, conclusions, and recommendations 1 1 this manuscript is a
companion paper to hutchings et al.(doi: 10.1139/a2012-011) and
vanderZwaag et al.(doi: 10.1139/a2012-013) also appearing in this issue.
these three papers comprise an edited version of a february 2012 royal
society of canada expert panel report. \emph{Environmental Reviews},
\emph{20}(4), 353--361.

\hypertarget{ref-jennings1999structural}{}
Jennings, S., Greenstreet, S., Reynolds, J., \& others. (1999).
Structural change in an exploited fish community: A consequence of
differential fishing effects on species with contrasting life histories.
\emph{Journal of Animal Ecology}, \emph{68}(3), 617--627.

\hypertarget{ref-jiao2009hierarchical}{}
Jiao, Y., Hayes, C., \& Cortés, E. (2009). Hierarchical bayesian
approach for population dynamics modelling of fish complexes without
species-specific data. \emph{ICES Journal of Marine Science: Journal Du
Conseil}, \emph{66}(2), 367--377.

\hypertarget{ref-jones2009operating}{}
Jones, M. L., Irwin, B. J., Hansen, G. J., Dawson, H. A., Treble, A. J.,
Liu, W., \ldots{} Bence, J. R. (2009). An operating model for the
integrated pest management of great lakes sea lampreys. \emph{The Open
Fish Science Journal}, \emph{2}, 59--73.

\hypertarget{ref-kallianiotis2004fish}{}
Kallianiotis, A., Vidoris, P., \& Sylaios, G. (2004). Fish species
assemblages and geographical sub-areas in the north aegean sea, greece.
\emph{Fisheries Research}, \emph{68}(1), 171--187.

\hypertarget{ref-kell2012robin}{}
Kell, L. T., \& De Bruyn, P. (2012). THE rOBIN hOOD aPPROACH fOR dATA
pOOR sTOCKS: AN eXAMPLE bASED oN aLBACORE. \emph{Collect. Vol. Sci. Pap.
ICCAT}, \emph{68}(1), 379--386.

\hypertarget{ref-kristensen2015tmb}{}
Kristensen, K., Nielsen, A., Berg, C. W., Skaug, H., \& Bell, B. (2015).
TMB: Automatic differentiation and laplace approximation. \emph{ArXiv
Preprint ArXiv:1509.00660}.

\hypertarget{ref-malick2015linking}{}
Malick, M. J., Cox, S. P., Mueter, F. J., Peterman, R. M., \& Bradford,
M. (2015). Linking phytoplankton phenology to salmon productivity along
a north--south gradient in the northeast pacific ocean. \emph{Canadian
Journal of Fisheries and Aquatic Sciences}, \emph{72}(5), 697--708.

\hypertarget{ref-maunder2015use}{}
Maunder, M. N., Deriso, R. B., \& Hanson, C. H. (2015). Use of
state-space population dynamics models in hypothesis testing: Advantages
over simple log-linear regressions for modeling survival, illustrated
with application to longfin smelt (spirinchus thaleichthys).
\emph{Fisheries Research}, \emph{164}, 102--111.

\hypertarget{ref-mantyniemi2009value}{}
Mäntyniemi, S., Kuikka, S., Rahikainen, M., Kell, L. T., \& Kaitala, V.
(2009). The value of information in fisheries management: North sea
herring as an example. \emph{ICES Journal of Marine Science: Journal Du
Conseil}, \emph{66}(10), 2278--2283.

\hypertarget{ref-mueter2006using}{}
Mueter, F. J., \& Megrey, B. A. (2006). Using multi-species surplus
production models to estimate ecosystem-level maximum sustainable
yields. \emph{Fisheries Research}, \emph{81}(2), 189--201.

\hypertarget{ref-mueter2002opposite}{}
Mueter, F. J., Peterman, R. M., \& Pyper, B. J. (2002). Opposite effects
of ocean temperature on survival rates of 120 stocks of pacific salmon
(oncorhynchus spp.) in northern and southern areas. \emph{Canadian
Journal of Fisheries and Aquatic Sciences}, \emph{59}(3), 456--463.

\hypertarget{ref-parkinson2004linking}{}
Parkinson, E. A., Post, J. R., \& Cox, S. P. (2004). Linking the
dynamics of harvest effort to recruitment dynamics in a multistock,
spatially structured fishery. \emph{Canadian Journal of Fisheries and
Aquatic Sciences}, \emph{61}(9), 1658--1670.

\hypertarget{ref-Pelc201556}{}
Pelc, R. A., Max, L. M., Norden, W., Roberts, S., Silverstein, R., \&
Wilding, S. R. (2015). Further action on bycatch could boost united
states fisheries performance. \emph{Marine Policy}, \emph{56}(0),
56--60.
\url{http://doi.org/http://dx.doi.org/10.1016/j.marpol.2015.02.002}

\hypertarget{ref-pestes2008bayesian}{}
Pestes, L. R., Peterman, R. M., Bradford, M. J., \& Wood, C. C. (2008).
Bayesian decision analysis for evaluating management options to promote
recovery of a depleted salmon population. \emph{Conservation Biology},
\emph{22}(2), 351--361.

\hypertarget{ref-peterman1999decision}{}
Peterman, R. M., \& Anderson, J. L. (1999). Decision analysis: A method
for taking uncertainties into account in risk-based decision making.
\emph{Human and Ecological Risk Assessment: An International Journal},
\emph{5}(2), 231--244.

\hypertarget{ref-punt2011among}{}
Punt, A. E., Smith, D. C., \& Smith, A. D. (2011). Among-stock
comparisons for improving stock assessments of data-poor stocks: The
``robin hood'' approach. \emph{ICES Journal of Marine Science: Journal
Du Conseil}, \emph{68}(5), 972--981.

\hypertarget{ref-ricker1958maximum}{}
Ricker, W. (1958). Maximum sustained yields from fluctuating
environments and mixed stocks. \emph{Journal of the Fisheries Board of
Canada}, \emph{15}(5), 991--1006.

\hypertarget{ref-ricker1972hereditary}{}
Ricker, W. (1972). Hereditary and environmental factors affecting
certain salmonid populations. \emph{The Stock Concept in Pacific
Salmon}, 19--160.

\hypertarget{ref-ricker1973two}{}
Ricker, W. (1973). Two mechanisms that make it impossible to maintain
peak-period yields from stocks of pacific salmon and other fishes.
\emph{Journal of the Fisheries Board of Canada}, \emph{30}(9),
1275--1286.

\hypertarget{ref-rokach2010ensemble}{}
Rokach, L. (2010). Ensemble-based classifiers. \emph{Artificial
Intelligence Review}, \emph{33}(1-2), 1--39.

\hypertarget{ref-safina2008study}{}
Safina, C., \& Lewison, R. L. (2008). Why study bycatch? An introduction
to the theme section on fisheries bycatch.

\hypertarget{ref-saila1983fishery}{}
Saila, S. B., \& Jones, C. (1983). \emph{Fishery science and the stock
concept}. University of Rhode Island.

\hypertarget{ref-sainsbury2000design}{}
Sainsbury, K. J., Punt, A. E., \& Smith, A. D. (2000). Design of
operational management strategies for achieving fishery ecosystem
objectives. \emph{ICES Journal of Marine Science: Journal Du Conseil},
\emph{57}(3), 731--741.

\hypertarget{ref-schindler2010population}{}
Schindler, D. E., Hilborn, R., Chasco, B., Boatright, C. P., Quinn, T.
P., Rogers, L. A., \& Webster, M. S. (2010). Population diversity and
the portfolio effect in an exploited species. \emph{Nature},
\emph{465}(7298), 609--612.

\hypertarget{ref-simon1972stock}{}
Simon, R. C., \& Larkin, P. A. (1972). \emph{The stock concept in
pacific salmon}. University of British Columbia.

\hypertarget{ref-su2004spatial}{}
Su, Z., Peterman, R. M., \& Haeseker, S. L. (2004). Spatial hierarchical
bayesian models for stock-recruitment analysis of pink salmon
(oncorhynchus gorbuscha). \emph{Canadian Journal of Fisheries and
Aquatic Sciences}, \emph{61}(12), 2471--2486.

\hypertarget{ref-sugihara1984ecosystems}{}
Sugihara, G., Garcia, S., Platt, T., Gulland, J., Rachor, E., Lawton,
J., \ldots{} others. (1984). Ecosystems dynamics. In \emph{Exploitation
of marine communities} (pp. 131--153). Springer.

\hypertarget{ref-team2009national}{}
Team, C. S. R. (2009). \emph{National conservation strategy for cultus
lake sockeye salmon (oncorhynchus nerka)}. Fisheries; Oceans Canada.

\hypertarget{ref-walters1999multispecies}{}
Walters, C. J., \& Bonfil, R. (1999). Multispecies spatial assessment
models for the british columbia groundfish trawl fishery. \emph{Canadian
Journal of Fisheries and Aquatic Sciences}, \emph{56}(4), 601--628.

\hypertarget{ref-walters2004fisheries}{}
Walters, C. J., \& Martell, S. J. (2004). \emph{Fisheries ecology and
management}. Princeton University Press.

\hypertarget{ref-zhou2010modified}{}
Zhou, S., Punt, A. E., Deng, R., Dichmont, C. M., Ye, Y., \& Bishop, J.
(2010). Modified hierarchical bayesian biomass dynamics models for
assessment of short-lived invertebrates: A comparison for tropical tiger
prawns. \emph{Marine and Freshwater Research}, \emph{60}(12),
1298--1308.

\end{document}