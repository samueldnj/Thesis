\documentclass[12pt,]{scrartcl}
\usepackage[letterpaper, margin = 1in, footskip = 0.25in]{geometry}
\usepackage{amsmath,amssymb,setspace,mathabx}
\usepackage{graphicx}
\usepackage{xcolor}
\usepackage{lscape}
\usepackage{verbatim}
\usepackage[left]{lineno}
\doublespacing

\providecommand{\tightlist}{%
  \setlength{\itemsep}{0pt}\setlength{\parskip}{0pt}}

% Redefine \includegraphics so that, unless explicit options are
% given, the image width will not exceed the width of the page.
% Images get their normal width if they fit onto the page, but
% are scaled down if they would overflow the margins.
\makeatletter
\def\ScaleIfNeeded{%
  \ifdim\Gin@nat@width>\linewidth
    \linewidth
  \else
    \Gin@nat@width
  \fi
}
\makeatother
\let\Oldincludegraphics\includegraphics
{%
 \catcode`\@=11\relax%
 \gdef\includegraphics{\@ifnextchar[{\Oldincludegraphics}{\Oldincludegraphics[width=\ScaleIfNeeded]}}%
}%

\setcounter{secnumdepth}{0}

\ifxetex
  \usepackage[setpagesize=false, % page size defined by xetex
              unicode=false, % unicode breaks when used with xetex
              xetex]{hyperref}
\else
  \usepackage[unicode=true]{hyperref}
\fi
\hypersetup{breaklinks=true,
            bookmarks=true,
            pdfauthor={Samuel D N Johnson},
            pdftitle={Assessment and Avoidance Management Tools for Multispecies Fisheries Constrained by Technical Interactions},
            colorlinks=true,
            citecolor=blue,
            urlcolor=blue,
            linkcolor=magenta,
            pdfborder={0 0 0}}
\urlstyle{same}  % don't use monospace font for urls


\title{Assessment and Avoidance Management Tools for Multispecies Fisheries
Constrained by Technical Interactions}
\subtitle{Thesis Proposal}
\author{Samuel D N Johnson}
\date{\today}


\begin{document}
\maketitle

\linenumbers

\abstract{The nature of commercial fishing gear makes it difficult to avoid
species with restrictive annual quotas while targeting highly productive
profitable species in multispecies fisheries. Exploitation of the high
productivity species is then constrained by the low quota species,
creating a pinch-point effect on fishery profitability. This thesis
proposes 2 management tools for improving profitability in the presence
of pinch-point effects. First, quota may be low because of uncertainty
about stock status owing to data limitations precluding up-to-date
assessments. Hierarchical multispecies assessment models that explicitly
acknowledge interactions among species may be used to extend assessments
to data-limited species and lift constraints. Second, if assessments are
unable to lift restrictive species quota improved avoidance tools may
alleviate pinch-point effects. By using machine learning algorithms on
commercial monitoring data it may be possible to redirect harvester
targeting to avoid restrictive species.}




\newpage

{
\hypersetup{linkcolor=black}
\setcounter{tocdepth}{3}
\tableofcontents
}
\section{Introduction}\label{introduction}

\subsection{Background}\label{background}

Sustainable management of any renewable resource requires an
understanding of system dynamics in response to exploitation. In a
multispecies fisheries context, the system is a collection of
semi-discrete self-sustaining fish populations or \emph{stocks} (Begg,
Friedland, \& Pearce, 1999) and the exploitation involves removing
individuals by fishing. Fishing effort impacts target species,
non-target species and fish habitat, and therefore a major challenge of
multispecies fishery management is to balance fishing yield with broader
sustainability goals.

Sustainable and scientifically defensible fishery management is built on
a foundation of fisheries stock assessment (Hilborn \& Walters, 1992).
Quantitative stock assessment methods combine elements of data science,
applied population ecology, risk assessment and resource management
(Figure 1). Analysts use data from multiple sources including scientific
surveys and commercial fishery monitoring to infer biological and
fishery dynamics and to characterise uncertainties and risks based on
these assessments. These inferences include estimates of species
abundance and productivity that are used to inform management decisions.

Stock assessments are lacking in most Canadian fisheries (Hutchings et
al., 2012), especially for non-target species. One reason is that
non-target species are typically of lower commercial importance, so
there is limited interest in assessments. More commonly, data
limitations preclude the assessment of certain species, known as
data-limited species. Surveys designed for data-moderate target species
are often unsuitable for non-target species and leave managers with the
choice of conducting a flawed assessment, or no assessment at all.

A lack of assessments for some species within a multispecies fishery
threatens sustainable management of the whole fishery in two ways.
First, a lack of assessments creates conservation risks by weakening the
link between management decisions and stock status, as the dynamic
nature of a fishery leads to more uncertainty about stock status as time
passes. Second, eco-certifiers typically require up-to-date stock
assessments for all species captured, regardless of whether those stocks
are targeted or not. A lack of eco-certification reduces market share of
a fishery both internationally and domestically, as buyers prefer
eco-certified products (Pelc et al., 2015).

\subsection{Assessments Acknowledging Technical
Interactions}\label{assessments-acknowledging-technical-interactions}

Stock assessments are traditionally performed for a single species at a
time, even though this approach may lead to sub-optimal outcomes for
multispecies fisheries (Gulland \& Garcia, 1984; Sugihara et al., 1984).
Sub-optimal outcomes may arise from not accounting for the effects of
interactions between species. In multispecies fisheries interactions are
one of two types: ecological or technical. Ecological interactions are
either non-trophic, such as competition, or trophic, between predator
and prey. Ecological interactions affect natural mortality of fish and
may bias estimates of species productivity when not taken into account
(Mueter \& Megrey, 2006). Technical interactions are caused by
non-selective fishing gear, and occur when multiple species are caught
simultaneously.

Within the single-species management paradigm, a species typically
comprises several distinct but interacting sub-stocks (Benson, Cox, \&
Cleary, 2015; Walters \& Martell, 2004). For example multiple
ecologically and technically interacting populations (i.e., stocks) of
Pacific salmon (\emph{Onchorynchus spp.}) species Chinook (\emph{O.
tshawtcha}), Chum (\emph{O. keta}), Coho (\emph{O. kisutch}), Pink
(\emph{O. gorbausch}), Sockeye (\emph{O. nerka}) and Steelhead (\emph{O.
mykiss}) occur along Canada's Pacific coast (Simon \& Larkin, 1972).
Each species is made up of genetically distinct subpopulations, defined
mainly by discrete spawning habitats and run timing that establish
quasi-isolated reproductive populations connected by low straying rates
(Ricker, 1972).

Managing hundreds of distinct fisheries is impractical (Walters \&
Martell, 2004) so salmon stocks are often grouped together into stock
complexes for management and assessment. For instance, in the Fraser
River, sub-populations of Chinook and Sockeye are grouped into aggregate
stock complexes called runs based on similarity in life history,
geographical locations of spawning habitat and arrival timing to
fisheries (DFO, 1999; English, Edgell, Bocking, Link, \& Raborn, 2011).
Managing Pacific salmon in runs has both advantages and disadvantages.
Aggregation leads to increased management efficiency and brings
statistical benefits from data pooling. However, to avoid overfishing of
some stocks complexes must be managed according to the weakest stock's
productivity (Figure 2) (Parkinson, Post, \& Cox, 2004; Ricker, 1958,
1973), as exemplified by the Cultus lake stock of Late run Fraser river
Sockeye salmon. Cultus lake Sockeye have had historic abundances of up
to 700,000 spawners but in 2004 fewer than 100 spawners returned from
the marine life phase, caused in part by previous over-harvesting at
average productivity for the Late run complex (Team, 2009). To avoid
continued over-harvesting, the entire Late run is now fished according
to the productivity of Cultus lake Sockeye.

The aggregate management schema used for Pacific salmon could be
modified and adopted in other multispecies fisheries. For example,
groundfish fisheries on the west coast of North America exploit stocks
of sablefish, Pacific halibut (\emph{Hippoglossus stenolopis}), several
species of rockfish (\emph{Sebastes spp.}), Pacific cod (\emph{Gadus
macrocephalus}), Dover sole (\emph{Microstomus pacificus}) and other
demersal species (Fisheries and Oceans, Canada, 2015). Different
groundfish genera and species have their own unique life histories and
reproductive strategies that respond differently to fishing pressure (S.
Jennings, Greenstreet, \& Reynolds, 1999). Different life histories and
reproductive strategies among groundfish imply different productivity
levels, similar to mixed-stock Pacific salmon fisheries.

Multiple interacting species with different productivity levels create
profitability constraints in multispecies fisheries managed through
quota systems (Baudron \& Fernandes, 2015; Hilborn, Punt, \& Orensanz,
2004). Constraints are caused by weaker, low productivity species that
cannot be avoided when targeting stronger, high productivity species.
Weaker species' quota is filled faster, so stronger species are
under-exploited in order to reduce the fishing pressure on the weakest
species, also known as pinch-point or choke species (Figure 3) (Hilborn
et al., 2004). An example of a pinch-point species is Boccacio rockfish
(\emph{S. paucispinis}) in the British Columbia groundfish fishery,
which are difficult to avoid when targeting lingcod (\emph{Ophiodon
elongatus}). Bocaccio rockfish are listed as Endangered by COSEWIC and
have a very low annual quota of around 110 metric tonnes (mt)\footnote{Committee
  on the Status of Endangered Wildlife in Canada.}, while lingcod are
highly productive with annual quota of around 3600mt. Avoidance of
Bocaccio by lingcod harvesters led to less than 33\% of Bocaccio quota
to be utilised between 2006 and 2014 (Figure 4). Technical interactions
between Bocaccio and lingcod means that this avoidance behaviour
resulted in a maximum of 25\% of lingcod quota being utilised in that
same time period (Figure 5). This underutilisation of quota translates
into a reduction of around \$10,000\footnote{Price per pound taken from
  California \emph{Status of the Fisheries Report}, 2008, assuming a
  parity conversion on average over the time period.} of gross annual
revenue to the fishery between 2006 and 2014.

\subsection{Assess and Avoid}\label{assess-and-avoid}

Profitability constraints caused by technical interactions may be
alleviated by conducting stock assessments of data-limited species and
avoiding pinch-point species. Species that lack up-to-date assessments
often have their quota set to a low level for conservation reasons,
creating artificial pinch-points. After assessment the quota of a
data-limited species can be scaled to a better estimate of stock-status
(Food and Agriculture Organization of the United Nations, 1995), which
may have two effects. Either the decrease in uncertainty about the stock
status allows removal of the pinch-point, or the pinch-point remains. In
the case where assessments show that the pinch-point cannot be removed
then an avoidance strategy is required.

One option for overcoming data limitations to assessments is by
explicitly acknowledging technical interactions in assessment models
(Mueter \& Megrey, 2006; A. E. Punt, Smith, \& Smith, 2011; Zhou et al.,
2010). Technical interactions can be acknowledged by aggregating
multiple species into the same assessment complex based on co-occurence
in fishing events, similar to Pacific salmon runs (Beverton et al.,
1984; Walters \& Martell, 2004). Statistical benefits of aggregation may
allow previously unassessed species to be assessed, and increase the
profitability of the fishery by relieving constraints and enabling
eco-certification. While more complicated than the single specie
paradigm, the benefit of assessing previously unassessed species may
outweigh the costs.

Figure 6 shows three possible models of fishery operation and
management. Models (a) and (b) are the current options for assessment in
multispecies fisheries. Model (a) is the status quo approach of single
species stock assessment, where every stock is treated as a separate
population (Hilborn \& Walters, 1992). Model (b) is the total
aggregation approach used for Pacific salmon (English et al., 2011),
where several species or stocks have their data combined and are then
assessed and managed as a single unit (Gaichas et al., 2012; Gulland \&
Garcia, 1984; Sugihara et al., 1984).

The total aggregation approach used by Pacific salmon may not be
suitable for assessing assemblages of multiple species with distinct
life histories and reproductive strategies. Model (c) in Figure 6
addresses this by keeping the data separate as in model (a), but
performs assessments for groups of stocks using statistical models that
link the data during estimation (Mueter \& Megrey, 2006; A. E. Punt et
al., 2011; Zhou et al., 2010).

In Chapters 1, 2 and 3 I will conduct a simulation study of hierarchical
stock assessment models that share data between species as Figure 6(c)
(Jiao, Hayes, \& Cortés, 2009; A. E. Punt et al., 2011; Zhou et al.,
2010). The statistical model assumes a hierarchical structure of
multispecies fisheries as shown in Figure 7, allowing for an an
intermediate level of between models shown in Figures 6(a) and 6(b).
Shared parameters in the hierarchical assessment model provide some of
the benefits of aggregation, but the separation of data streams allows
for species specific estimates of abundance and productivity (Jiao et
al., 2009).

The focus of Chapter 1 is to create a simulation-estimation procedure to
study hierarchical assessment models for multi-species assemblages with
no sub-stock structure. Data generated by a process error population
dynamics model and observation model are provided to hierarchical
estimators (A. E. Punt et al., 2011; Zhou et al., 2010). The statistical
performance of the estimators is then quantified by comparing the true
values of parameters to estimated values.

In Chapter 2, the simulation-estimation procedure of Chapter 1 is
extended to include a spatial sub-stock structure for each species
(Figure 9). Including multiple sub-stocks increases the resolution of
the data and allows for multiple stock specific life history parameter
values within each species (Su, Peterman, \& Haeseker, 2004), however
challenges may arise from disaggregation of species data across multiple
spatial strata. Bias and precision are estimated and compared to
coastwide model bias and precision, to analyse the benefits and costs of
including increased structure in the model.

In Chapter 3, a closed loop feedback simulation framework is used to
evaluate management procedures using spatially structured multistock and
multispecies hierarchical models (Figure 10). This requires an operating
model that simulates population dynamics of multiple interacting fish
species, uncertain observations made by scientific surveys and effort
dynamics of multiple fishing fleets with different gear types exploiting
each population (Clark, 2010; Hilborn \& Walters, 1987; M. L. Jones et
al., 2009; Walters \& Bonfil, 1999). Uncertain data provided by the
operating model react with the management procedure to produce complex
emergent properties. Closed loop simulation offers a low-stakes option
for analysing those properties and the associated risks.

In Chapter 4, I will investigate a data-based approach to avoiding
non-target species and estimate its economic value. Reliable, spatially
explicit commercial data is becoming more abundant with increasing
observer coverage in modern fisheries. Concurrent with this, statistical
learning methods are emerging that allow for analysis of data that isn't
collected under strict experimental designs (Hastie, Tibshirani, \&
Friedman, 2009; Lennert-Cody \& Berk, 2007), such as commercial fishing
data.

\subsection{Study System}\label{study-system}

British Columbia's Groundfish Fishery (Fisheries and Oceans, Canada,
2015) is a group of 7 fisheries that spatially and temporally overlap on
the BC coast. The overlapping fisheries are managed across 8 statistical
areas (Figure 11) by one integrated individual transferrable quota
system, allowing temporary and permanent transfers of quota allocations
between licenses in different fleets. All catch and discards are
deducted from quota allocations, and are therefore monitored on 100\% of
vessels by at sea observer or electronic monitoring systems. Skippers
who exceed their quota share must either obtain more from other
harvesters, or stop fishing for the season.

Integrated management of the British Columbia groundfish fishery creates
pinch-points on quota utilisation, caused by technical interactions
between high value target species and data limited non-target species.
Many species lack up-to-date assessments creating, to varying degrees,
pinch-point effects that may be alleviated by assessing and avoiding
those species (Driscoll, 2014).

In Chapters 1, 2 and 3 the simulation study will use a multispecies
complex composed of all flatfish except halibut in the British Columbia
groundfish fishery as the biological component of the operating model.
The complex is made up of \textbf{D}over sole (\emph{Microstomus
pacificus}), \textbf{E}nglish sole (\emph{Parophrys vetulus}),
\textbf{R}ock sole (\emph{Lepidopsetta bilineata}), \textbf{P}etrale
sole and \textbf{A}rrowtooth flounder (\emph{Atheresthes stomias})
(Fisheries and Oceans, Canada, 2015), and called the \textbf{DERPA}
complex for brevity. All members of DERPA are from the family
\emph{Pleuronectidae} of right-eyed flounders, making DERPA suitable for
a hierarchical approach due to similar but distinct life and
evolutionary histories. Furthermore, Dover sole, Petrale sole and
Arrowtooth flounder are subject to technical interactions, as they often
co-occur in fishing gear that encounters Sablefish (Figure 12). Halibut
are excluded as they are managed by a separate trans-boundary authority.

The amount of data available for DERPA flatfish species varies, and so
does the timing of stock status assessments. Rock sole was assessed in
2016 (K. R. Holt, Starr, Haigh, \& Krishka, 2016) and 2014 (DFO, 2014),
and Arrowtooth flounder in 2015 (DFO, 2015), but before that both
species were not assessed for close to a decade (Jeff Fargo \& Starr,
2001; Jeff Fargo, Kronlund, Schnute, \& Haigh, 2000).\footnote{Rock sole
  were assessed in 2005 in an unpublished working paper, see K. R. Holt
  et al. (2016).} English and Petrale sole were last assessed in 2009
(Starr, 2009a, 2009b). Dover sole was last assessed in 1999 and has
never been assessed using a model based assessment (J. Fargo, 1999).

In a departure from the study system in Chapters 1, 2 and 3, in Chapter
4 I will use computational methods to forecast the presence of sub-legal
sized sablefish in fishing events. Sablefish are at historic low
abundances and are subject to a rebuilding strategy (S. Cox \& Kronlund,
2009), and the reduction of discard induced mortality has been
identified as a means to increase sablefish spawning stock biomass
without necessarily reducing quota (S. Cox, Kronlund, \& Lacko, 2011).
Discarding of legal-sized sablefish (\textgreater{}55cm, good condition)
is disincentivised by a quota deduction adjusted for discard induced
mortality, but no incentive structure exists for unmarketable sablefish
(\textless{}55cm, poor condition). This incentive structure is evident
in the distribution of sablefish discarding, with an average of 70\% of
sablefish discards in the trawl sector between 1997 and 2006 made up by
sub-legal sized fish\footnote{From the PacHarvTrawl database housed at
  the Fisheries and Oceans, Canada, Pacific Biological Station, Nanaimo.}.

\section{Chapter 1: Estimating Coastwide Abundance and Productivity in a
Multispecies Groundfish Fishery via a Hierarchical Stock Assessment
Model}\label{chapter-1-estimating-coastwide-abundance-and-productivity-in-a-multispecies-groundfish-fishery-via-a-hierarchical-stock-assessment-model}

\subsection{Background}\label{background-1}

Quantitative stock assessment models incorporate population dynamics
processes (Figure 1.1), observational data (Figure 1.2) into a
statistical model (Figure 1.3) (Hilborn \& Walters, 1992). Model inputs
are candidate parameter values that are compated to data in the
statistical model to produce posterior density or likelihood function
values as outputs. Statistical model output is then optimised or
integrated over the input parameters to extend inferences about stock
productivity and status in the form of distributional estimates.

Hierarchical statistical models are becoming increasingly popular for
analysing complex fisheries data. In Pacific salmon stock and
recruitment analyses, both Bayesian and frequentist (mixed effects)
hierarchical models are used in meta-analyses of multistock populations
(Malick, Cox, Mueter, Peterman, \& Bradford, 2015; Su et al., 2004).
More related to this thesis, stock assessment models that use
hierarchical statistical models are sometimes used to assess
multispecies complexes where data limitations are an issue for single
species management, such as technical interactions between data-limited
species (A. E. Punt et al., 2011) or difficulties in species
identification (Jiao et al., 2009).

In this chapter, I will use a simulation-estimation procedure to study
hierarchical Bayesian (Zhou et al., 2010) and frequentist (A. E. Punt et
al., 2011) state space multispecies assessment models. The multispecies
models are used to simultaneously assess a simulated version of the
DERPA complex of flatfish. In a comparison between single species and
hierarchical models applied multispecies groups including data-limited
species, it has been shown that the hierarchical models induce a change
in parameter estimates for data-limited species (Kell \& De Bruyn, 2012;
A. E. Punt et al., 2011). However, it is unknown if that change is an
increase or decrease in bias. As a result, for this paper I address the
following research question.

\textbf{QUESTION:} How do estimates of unfished biomass \(B_0\), growth
\(r\) and catchability \(q\) made by hierarchical multispecies models
compare to estimates from single species models?

Simulated scientific and commercial data are used to test hierarchical
assessment models. True parameter values used for simulation can be
compared to estimated parameters in Monte-Carlo trials to understand
bias and precision of both estimators. Estimators are then tested across
a range of scenarios representing implications of technical interactions
between species, and contrasts in data availability.

\subsection{Methods}\label{methods}

Each species in the DERPA complex will be simulated using the model
defined in Table 2. Population dynamics are simulated by a simple
biomass dynamics process error-model (Figure 1.1, Table 2 equations
T2.2, T2.4), fishery dependent catch is generated using fishing
mortality as an input (Eq T2.3) and fishery independent observations of
catch per unit effort (CPUE) are generated by the observation model
(Figure 1.2, Eq T2.5).

Multispecies data produced by the simulation model will be supplied to
both a Bayesian and frequentist version of a hierarchical state-space
assessment model (Figure 7). Both assessment models are specified in the
same way, shown in Table 3. The difference between the models is in how
the inferences are extended. For the Bayesian state space model the
posterior density (Table 3, Eq T3.8) is integrated over all parameters
included in \(\Theta\) (Eq T3.2) to produce marginal distributions for
each parameter (Gelman, Carlin, Stern, \& Rubin, 2014). For the
frequentist state space model, also known as a random effects model, the
posterior density is integrated over random effects (process errors) and
prior distributions to produce a marginal ``true'' likelihood, which is
then maximised as in traditional likelihood methods (de Valpine \&
Hastings, 2002).

Both models require an integration method to produce marginal
distributions or likelihoods (de Valpine \& Hastings, 2002; Gelman et
al., 2014, Maunder, Deriso, \& Hanson (2015)). Integration generally
requires numerical methods like Markov-Chain Monte Carlo (MCMC)
algorithms for distribution sampling of complex non-linear, non-Gaussian
statistical models. To this end, the Bayesian model is coded using the
Automatic Differentiation Model Builder (ADMB) suite (Fournier et al.,
2012) and the random effects model using Template Model Builder
(Kristensen, Nielsen, Berg, Skaug, \& Bell, 2015). Both software
packages provide fast numerical integration to produce marginal
distributions, with TMB being developed specifically for models
utilising a large number of random effects.

Model testing proceeds through four experimental scenarios that modify
simulation input parameters representing multispecies interactions and
data limitations. Parameter estimates from each trial are then compared
to their true values generated by the simulator to estimate bias and
precision of the models in each scenario.

The first two scenarios investigate model assumptions about process
error deviations \(\epsilon_{t}, \zeta_t\) and species catchability
coefficients \(q_s\). Both parameters are representive of interactions
between species in the complex. For example, species that share the same
habitat will encounter the same environmental variation, reflected in
coastwide process error deviations \(\epsilon_t\) (A. E. Punt et al.,
2011). Moreover, interactions between each of the species may cause
correlations in their species speficic process errors \(\zeta_t\),
reflected in the covariane matrix \(\Sigma\). Similarly, species that
are fished by the same gear may have similar interactions with fishing
gear leading to correlations in catchability \(q_s\).

Shared priors are defined for process error deviations (Eqs T3.7, T3.8)
and catchability parameters (Eq T3.9). Bias and precision are measured
for a range of fixed values of the prior variance
(\(\sigma^2, \kappa^2 \in (0,\infty)\)) (Gelman et al., 2014, Ch 5.5)
and multiple configurations of the covariance matrix \(\Sigma\).

The remaining two scenarios contrast information available from survey
observations and resource responses to exploitation pressure.
Observation error is a direct measurement of the quality of data
obtained by scientific surveys, so contrasts in observation error
variance \(\tau_s^2\) simulate differing levels of data availability
between species in an assemblage. Fishery development histories,
characterised by fishing mortality \(F_{s,t}\) trajectories (Figure 8),
are a source of information based on the way a fish population responds
to changes in fishing pressure (Hilborn \& Walters, 1992, Ch 2).

\subsection{Expected Results}\label{expected-results}

I expect this chapter to result in a working knowledge of how
hierarchical stock assessment models change the estimates of abundance
and productivity when applied to multispecies assemblages. Estimates of
model bias and precision as functions of correlation strengths,
observation error variance and historical fishing are produced. Results
are to be published in a paper about the statistical properties of 2
hierarchical multispecies assessment models.

Assumptions about the strength of correlations in shared parameters are
likely to introduce bias through shrinkage towards a mean (Mueter,
Peterman, \& Pyper, 2002). The extent of the shrinkage introduced can be
understood by producing bias and precision estimates under a range of
fixed values of shared prior variance.

The extent to which limitations on data and species specific information
can be overcome (A. E. Punt et al., 2011), if at all, can be quantified
through bias and precision estimates resulting from scenarios
contrasting data-availability and fishing histories. This is especially
helpful for fisheries in which there are limited historical fishing and
scientific data available, or limited resources for improving existing
scientific surveys.

\section{Chapter 2: Adding Spatial Multistock Structure to Multispecies
Hierarchical Stock Assessment
Models}\label{chapter-2-adding-spatial-multistock-structure-to-multispecies-hierarchical-stock-assessment-models}

\subsection{Background}\label{background-2}

A high degree of spatial variation in genetics, morphology, life-history
and behaviour is apparent in many exploited fish populations (Hilborn,
Quinn, Schindler, \& Rogers, 2003; Schindler et al., 2010). Management
of exploited fishes without acknowledgement of this variation risks
eroding biodiversity and increasing species vulnerability to
environmental variation (Benson et al., 2015; Cope \& Punt, 2011;
Hilborn et al., 2003).

Aggregation of sub-stocks into a single management unit over large
spatial scales relies on migration to mitigate localised depletion of
discrete substocks. The assumption is that despite spatial
disaggregation of the stock, sub-stocks are connected by migration
creating a rescue effect (Dulvy, Sadovy, \& Reynolds, 2003). Rescue
effects are then believed to reduce the risks of managing spatially
complex species in a single aggregate (Cope \& Punt, 2011). However,
this rescue effect is highly dependent on dispersal and recruitment
patterns in the meta-population and individual natural mortality rates
of sub-stocks (Benson et al., 2015).

When stock structure is easily identified, as with Pacific salmon, there
are advantages to managing a species at the level of individual stocks.
For example, by estimating productivity levels for 43 individual stocks
of Pink salmon the effects of local variation in sea surface temperature
could be discovered (Su et al., 2004). Furthermore, estimating
individual productivity levels within a management complex reduces the
risk of overfishing weak stocks whose productivity is less than the
aggregate's (Figure 2).

Managing multistock populations also has its challenges. When the exact
nature and connectedness of the spatial stock structure is unknown, it
is unclear whether or not aggregation is the more precautionary
management approach (Benson et al., 2015). Furthermore, for a
data-limited species further disaggregation of the data will only
deplete the quantity of data available in each strata at the finer
resolution, raising further barriers to assessment.

A hierarchical stock assessment model may overcome data limitations from
disaggregation when managing for multiple stocks in a multispecies
fishery (A. E. Punt et al., 2011). Life histories within species are
likely to be similar, allowing for prior distributions on life history
parameters that are shared between stocks. Similarly, sub-stocks of
multiple species share habitat and experience the same environmental
variation, allowing for a local spatial effect on process error
(Kallianiotis, Vidoris, \& Sylaios, 2004). To investigate these effects
I ask the following research question.

\textbf{QUESTION:} How do estimates of abundance and productivity in a
multistock, multispecies hierarchical model compare to those of a
coastwide multispecies hierarchical model?

To answer this question in a simulation study I use the DERPA complex,
which exhibits evidence of sub-stock structure. For example, the species
population of English sole on the British Columbia coast is managed as
two segregated major stocks with limited migration (Hart, Clemens, \&
others, 1973). Simulated data from a multi-stock model of the DERPA
complex is provided to both a coastwide and multistock hierarchical
multispecies model. Both models produce parameter estimates, and bias
and precision are compared.

\subsection{Methods}\label{methods-1}

The DERPA complex will be simulated as individual stocks \(p\) of each
species \(s\), with stocks corresponding to the discrete populations
identified in stock previous stock assessments (DFO, 2015; J. Fargo,
1999; K. R. Holt et al., 2016; Starr, 2009b, 2009a) (Figure 11). The
model used for each stock, defined in table 5, is a process error
surplus production model with a term representing migration between
substocks of the same species. Migration from substock \(p\) to substock
\(p'\) within species \(s\) is possible with net migration rate
\(\phi_{s,p,p'}\), making stock population dynamics interdependent (Tabl
5, Eq T5.4).

Population dynamics are affected by environmental process errors with
three components. The first component \(\epsilon_t\) affects all
populations identically. The second component \(\zeta_t\) affects stocks
within species identically, and between species according to the
covariance matrix \(\Sigma^{(S)}\). Finally, the third component
\(\xi_t\) is stock specific, with draws correlated according to the
covarance matrix \(\Sigma^{(P)}\). The stock specific component is meant
to capture spatial covariation between stocks of different species that
share the same habitat.

The multistock estimation procedure will use three layers of
hierarchical structure to include multiple species, each containing
multiple stocks (Table 6, Figure 9). The multiple stocks within each
species share prior distributions on growth \(r\) and catchability \(q\)
parameters at the species level (Figure 9(b); Eqs T6.7, T6.8). The
multistock prior mean catchabilities at the species level then share a
multispecies prior (Figure 9(c); Eq T6.10). Additionally, the process
error components are shared at the appropriate level (Eqs T6.9, T6.11,
T6.12).

Five experimental scenarios are proposed to evaluate bias and precision
of the multistock estimator as functions of data quality contrasts,
fishery development history and covariation due to shared environment.
Four scenarios are extended from Chapter 1 to account for increased
depth in the assemblage structure, including covaration between species
in \(\Sigma^{(S)}\) and covariation between stocks in \(\Sigma^{(P)}\).
The additional scenario models increased data-limitation introduced by
disaggregating an already data-limited species into multiple stocks.
Disaggregation could lead to increased observation error variance or
entirely missing observations for some stocks.

Finally, the multistock estimator will be compared to the coastwide
estimator in Table 3. The coastwide model uses aggregated data from the
multistock simulator, and the 5 scenarios of the previous paragraph are
repeated. Bias and precision will be recorded and compared between
estimators.

\subsection{Expected Results}\label{expected-results-1}

I expect this chapter to deepen understanding of hierarchical estimators
and their application in a multistock context. Adding stock structure
involves increased model complexity and reduced data availability due to
disaggregation, introducing a tradeoff. This tradeoff is then evaluated
by varying data availability and model complexity and examining how
model bias and precision change for the coastwide and multistock models.
A publication detailing the tradeoffs between bias and precision under
different model structures is expected to result from this analysis.

\section{Chapter 3: Management Performance of Hierarchical Multispecies
Assessment
Models}\label{chapter-3-management-performance-of-hierarchical-multispecies-assessment-models}

\subsection{Background}\label{background-3}

Fisheries management procedures extend beyond the stock assessment model
in practice (Figure 1). Stock assessment output (Figure 1.3) informs a
decision rule (Figure 1.4) that determines the amount of fishing effort
expended to collect the harvest quota (Figure 1.5). This effort
dynamically influences fish populations and their habitat (Figure 1.1),
providing feedback in the form of new data (Figure 1.2) that is used for
assessment.

An important test for an assessment model is how it performs as part of
a feedback management procedure. Management procedures include harvest
strategies, which are limits on catch or effort in the fishery, and
decision rules that scale controls to the health of the stock, such as a
harvest rate (Hilborn \& Walters, 1992, Ch. 15). Management procedures
made up of decision rules, harvest strategies and assessment models
represent a season in the management of a fishery.

In this chapter, I will use closed loop simulation to evaluate
management procedures based on hierarchical multispecies stock
assessment models. Closed loop simulation modeling explicitly quantifies
feedback in a dynamic system (de la Mare, 1998; Sainsbury, Punt, \&
Smith, 2000). In a fisheries management context, the closed loop
includes the management procedure, fish stocks and commercial and
scientific data in a feedback loop (Figure 9). The fishery, population
dynamics and scientific survey are part an operating model (M. L. Jones
et al., 2009) that provides data to the assessment model and harvest
control rule as part of a management procedure. Management procedure
evaluation then proceeds by experimentally adjusting operating model and
assessment model parameters and observing the emergent behaviour. In
this way, potential risks of management can be quantified under a given
set of assumptions.

Realistic predictions about management procedure performance require a
complex operating model that can accurately reflect fishery history.
Historical exploitation patterns are dependent on the spatial
distribution of fishing effort, induced by the targeting behaviour of
harvesters (Hilborn \& Walters, 1987; Walters \& Bonfil, 1999; Walters
\& Martell, 2004 Ch. 9.3). Targeting behaviour is dependent on several
factors, including catch composition and expected financial reward, and
can be simulated by including a fishing effort dynamics model for
multiple fishing fleets (gear types) in the operating model. Effort
dynamics are based on fishery dependent catchability parameters
\(q_{f,s,t}\) (Table 7), which can be empirically estimated from
commercial data or parametrically simulated. Using this framework, I
address the following research question.

\textbf{QUESTION:} How do multispecies hierarchical assessment models
perform when simulaneously managing multiple target and non-target
species?

I will answer this question by running closed loop simulations of the
DERPA complex under different management and ecological scenarios. A
validated operating model that makes use of historical fishery effort
and observed population dynamics is used to simulate management
procedures forward in time and assess risks of future management
decisions. Risks of assessment model errors, harvest control rules and
effort dynamics can be tested across multiple experimental operating
model scenarios and management. Experiments include contrasts in
data-quality between species, spatial aggregation of multistock
structure, covariation due to environmental forcing (Dichmont, Deng,
Punt, Venables, \& Haddon, 2006), and changes in effort dynamics driven
by economic forces..

\subsection{Methods}\label{methods-2}

The closed loop simulator of the DERPA complex requires an operating
model that includes effort dynamics (Table 7; M. L. Jones et al., 2009).
At each time step \(t\), the current state of each fish stock or species
is estimated by the assessment model. Asessment models then forecast
abundance at time \(t+1\), which is passed through a harvest control
rule (HCR) to generate a total allowable catch (TAC) for each species.
The TAC for each species is then supplied to the operating model, which
distributes fishing effort across the space in order to maximise some
objective, such as profit, subject to the constraints of the TAC.

Four classes of model are available for simulating short term
distribution of fishing effort (Walters \& Martell, 2004, Ch. 9.3). From
least to most complex the four classes are: (i) gravity models (Walters
\& Bonfil, 1999), (ii) ideal free distribution (IFD) models (Benson et
al., 2015), (iii) sequential effort allocation models (Hilborn \&
Walters, 1987) and (iv) individual based models.

For spatial allocation of fishing effort I use a simplified IFD model
(Walters \& Bonfil, 1999) with a numerical effort response model for
fish vulnerability (Cox \& Walters, 2002). The IFD model is chosen
because of the large spatial scale of discrete stocks in the DERPA
complex (Figure 11), allowing for the more complex IFD model over the
simplified gravity model more suited to finer resolution. The numerical
effort response model allows for the transition of individuals to and
from a vulnerable state, reflecting the reality that not all habitat can
be fished by all gear.

External economic forces are included in the effort dynamics model. The
IFD model ranks the quality of each fishing site by the profitability
\(pr_i\) of fishing at site \(i\). Profitability is a function of fixed
and variable fishing costs, ex-vessel sale price of catch and the cost
to acquire the necessary quota for bycatch. Quota prices are subject to
market forces, such as scarcity, meaning bycatch quota for pinch-point
species can at times exceed the ex-vessel sale price of that species,
decreasing the expected profitability of a given site and affecting
harvester behaviour.

The closed loop simulation tests future performance of management
procedures using single species, coastwide multispecies and multistock
multispecies models in experimental scenarios. Experiments test a range
of observation error variances, process error variances, fishery
development history and correlations in catchability \(q_{s,j,t}\). Each
simulation measures quota utilisation, species depletion, probability of
exceeding optimal instantaneous fishing mortality and annual average
variation. Simulation output is then used to compare between scenarios
and management procedures, quantifying performance and risks of each
procedure.

\subsection{Expected Results}\label{expected-results-2}

This chapter is expected to result in an understanding of how
hierarchical management procedures perform in multispecies fisheries.
Performance of both coastwide and multistock models is compared in
closed loop simulation against the status quo management of the DERPA
complex, which involves intermittent assessments at best. Furthermore,
simulations of the DERPA complex that model status quo management may
uncover risks not considered in the current management system.

Results will be published in two articles in the primary literature. The
first article will be detail the model linking complex market forces to
IFD effort dynamics simulator. The second article will publish the
operating model and the pinch-point dynamics that are expected to emerge
from the closed loop simulation.

\section{Chapter 4: Avoiding non-target
species.}\label{chapter-4-avoiding-non-target-species.}

\subsection{Introduction}\label{introduction-1}

Quota on target species in the British Columbia groundfish fishery is
increasingly constrained by not only restrictive quotas on pinch-point
species, but size limits on target species. For example, an average of
160 tonnes per year of Sablefish below 55cm in length were discarded due
to size regulations by trap and trawl fishing vessels between 2007 and
2015 in the fishery\footnote{From the Groundfish Fishery Operating
  System (GFFOS), Pacific Biological Station, Nanaimo.}. These
undersized individuals represent potential growth and recruitment
overfishing of the sablefish resource, constraining future TACs by
lowering species productivity (S. Cox et al., 2011).

There is general agreement in the literature that incidental catch and
discarding should be reduced as much as possible or practical (Crowder
\& Murawski, 1998; Pelc et al., 2015; Safina \& Lewison, 2008; Saila \&
Jones, 1983). Mortality of immature individuals caused by unregulated or
regulated (i.e.~size, quota, trip limits) discarding contributes to both
recruitment and growth overfishing of fish stocks (Crowder \& Murawski,
1998). Furthermore, bycatch has an impact on the ecosystem containing
the target resource, including all non-resource species and habitats
that interact with the fishing gear (Safina \& Lewison, 2008).

In this chapter I will test the feasibility of using a model-based
approach to predicting fishing events that encounter and discard
juvenile sablefish by analysing commercial fishing data for the purposes
of avoiding regulatory discarding. Data generated by commercial fishing
is not randomly sampled, so traditional statistical models that rely on
the central limit theorem are unsuitable. Instead, novel statistical
learning models that may sidestep these restrictions are used to search
for correlations in commercial data (C. M. Bishop, 2006; Hastie et al.,
2009).

Statistical learning models are trained and optimised on commercial data
from the British Columbia groundfish fishery to classify presence and
absence of juvenile sablefish for a given fishing event. Event
predictions from all models are then combined into a multi-model, or
ensemble, classifier (Rokach, 2010; Vrbik \& McNicholas, 2015). Ensemble
classifiers use weighted model averaging techniques to overcome
potential overfitting to the training data. A set of quarantined data is
then used to test the performance of the classifiers using multiple
metrics (Freeman \& Moisen, 2008).

The feasibility of a tool to avoid regulatory discarding requires an
economic benefit to harvesters. Because juvenile Sablefish are discarded
under size regulations, no discard induced mortality is deducted from
harvester quota (Fisheries and Oceans, Canada, 2015). No reduction in
quota implies a lack of economic incentive for harvesters to avoid
conditions that lead to the catch of juvenile fish.

\textbf{QUESTION:} What is the benefit of using a machine learning
approach for predicting the presence and absence of juvenile sablefish
in commercial fishing events, compared to the status quo?

Economic benefit of the ensemble classifier is measured by estimating
the value of information provided by the classifier (Mäntyniemi, Kuikka,
Rahikainen, Kell, \& Kaitala, 2009). Classifier performance is combined
with empirical estimates of the probability of encounter in a decision
analysis, with utility provided by a dollar value based on the costs and
benefits of successful and unsuccessful avoidance.

\subsection{Methods}\label{methods-3}

Predictive capacity of ensemble classifiers to detect juvenile Sablefish
is tested on 4 sets of commercial fishing data from the British Columbia
groundfish fishery. The fishery data is contained in the Groundfish
Fishery Operating System (GFFOS) data base that contains spatially and
temporally explicit data for every fishing event in the fishery since
2005. The 4 data sets are split by gear type, with data sets containing
events using (i) trawl only, (ii) longline trap only, (iii) longline
hook only and (iv) all gear types.

For each data set, two types of classifiers are developed. First, a
finite mixture model will be tested (Frühwirth-Schnatter, 2006). Finite
mixture models are weighted combinations of single statistical models,
allowing for highly irregular data to be modeled using mixtures of
parametric distributions or explanatory or descriptive models. Second,
an ensemble classifier using modern machine learning, or big data,
techniques will be developed using Random Forest (Breiman, 2001), Naive
Bayes (Meyer, Dimitriadou, Hornik, Weingessel, \& Leisch, 2015) and
Artificial Neural Network (W. N. Venables \& Ripley, 2002) classifiers.

Component model configurations for both types are chosen based on
average performance over Monte-Carlo trials of a validation procedure
(Hastie et al., 2009, Ch 7.2). Performance of classifiers is measured
using multiple metrics including percentage correctly classified, area
under receiver operating characteristic curves, precision and recall
(Freeman \& Moisen, 2008). Component classifiers that perform the best
are then combined, with the configurations depending on the type.

Finite mixture models are combined in a prescribed way with weights
estimated from the data, but machine learning classifiers have some more
flexibility. Methods range from simple weighted model stacking to
Bayesian Model combination (Rokach, 2010). Ensemble classifiers are then
tested on a reserved portion of the data to estimate the classification
error rate of the ensemble.

A formal decision analysis is performed to estimate the value of
information provided by using each classifier on each data set
(Mäntyniemi et al., 2009; Pestes, Peterman, Bradford, \& Wood, 2008;
Peterman \& Anderson, 1999). Classifiers are included in the analysis as
a form of expert opinion, adjusting the probability of encountering
juvenile Sablefish, based querying the expert prior to a fishing event.
The value of information is then the difference in the expected utility
of fishing with the classifier's help and the expected utility of
fishing without it. The utility is dependent on the costs setting gear,
sorting discards from the catch, as well as the value of landed catch.

\subsection{Expected Results}\label{expected-results-3}

It is expected that statistical learning will be economically feasible
for the avoidance of non-target species. However, the net benefit is
will likely depend on the nature of the species being avoided. For
example, pinch-point species with restrictive quota that is costly to
acquire, such as Yelloweye rockfish, may result in a greater net
benefit, while regulatory discards of juvenile individuals with no quota
penalty may result in a lesser net benefit. This could be overcome by
including the avoidance technology in larger closed loop simulations of
the management system, and seeking predictions of a long term benefit in
the form of higher TACs of mature sablefish.

Challenges in this chapter include acquiring spatially and temporally
explicit data for bycatch of juvenile sablefish, and estimating the
costs and benefits for the decision analysis. For data acquisition, the
privacy act creates a limit on the resolution of commercial data,
requiring creativity an perspiration in choosing an aggregation scale.
Estimates of the costs and benefits of fishing may exist in the
literature for other fisheries, but this may be best informed by asking
skippers directly.

\section{Conclusion}\label{conclusion}

This thesis is a study of assessment and avoidance tools that may
improve management of integrated multispecies fisheries, in which
technical interactions cause constraints on fishery profitability.
Profitability is constrained when the effort targeting directed, high
value species encounters non-target species with restrictive quota.
Restrictive species quota may be caused by data limitations precluding
regular assessments, or conservation concerns requiring rebuilding
strategies. In either case, those species become pinch-points on the
efficient management of the fishery.

Hierarchical assessment models studied in Chapters 1, 2 and 3 may
overcome data limitations and allow assessments to be extended to
species that were previously unassessed. Extending assessments to
previously unassessed species may or may not relieve pinch points by
reducing uncertainty about stock status, but will always increase
scientific defensibilty. Indeed, up-to-date and regular assessments of
non-target species allows for improved ratings by eco-certification
bodies (Driscoll, 2014). Improved eco-certification can create benefits
by improving access to foreign and domestic markets.

Closed loop simulations of hierarchical assessment models studied in
Chapter 3 may have further benefits in multispecies fishery management,
specifically in improving the allocation of scientific resources. By
assessing groups of multiple species with similar life and evolutionary
histories, it may be possible to take biological samples more
efficiently. For example, age and length sampling may occur only for
higher value species in a group, with lower value species sampled for
length only. Then length and age can be related through a shared
multispecies prior defined in the hierarchical assessment model. If
model stability is an issue, low frequency age sampling of the lower
value species may be necessary. Closed loop simulation can assess the
potential risks associated with these and other survey design
modifications.

Avoidance techniques are necessary when assessment methods are unable to
relieve pinch-point effects of low quota species. The statistical
learning methods to be studied in Chapter 4 are a novel approach to the
avoidance problem, combining technological and fleet communication
approaches. A centralised communication system can use reported observer
data to provide near-real-time information to harvesters, detailing the
probability of non-target species encounter under given conditions. The
system is not unique to juvenile Sablefish and could be extended to any
non-target species encountered, which would change the expected net
economic benefit of the product.

\section*{References}
\setlength{\parindent}{-0.2in}
\singlespacing
\small
\setlength{\leftskip}{0.2in}
\setlength{\parskip}{8pt}
\vspace*{-0.4in}
\noindent

\hypertarget{refs}{}
\hypertarget{ref-baudron2015adverse}{}
Baudron, A. R., \& Fernandes, P. G. (2015). Adverse consequences of
stock recovery: European hake, a new ``choke'' species under a discard
ban? \emph{Fish and Fisheries}, \emph{16}(4), 563--575.

\hypertarget{ref-begg1999stock}{}
Begg, G. A., Friedland, K. D., \& Pearce, J. B. (1999). Stock
identification and its role in stock assessment and fisheries
management: An overview. \emph{Fisheries Research}, \emph{43}(1), 1--8.

\hypertarget{ref-benson2015evaluating}{}
Benson, A. J., Cox, S. P., \& Cleary, J. S. (2015). Evaluating the
conservation risks of aggregate harvest management in a
spatially-structured herring fishery. \emph{Fisheries Research},
\emph{167}, 101--113.

\hypertarget{ref-beverton1984dynamics}{}
Beverton, R., Cooke, J., Policansky, D., Csirke, J., Roughgarden, J.,
Doyle, R., \ldots{} others. (1984). Dynamics of single species. In
\emph{Exploitation of marine communities} (pp. 13--58). Springer.

\hypertarget{ref-bishop2006pattern}{}
Bishop, C. M. (2006). \emph{Pattern recognition and machine learning}
(Vol. 4). springer New York.

\hypertarget{ref-breiman2001random}{}
Breiman, L. (2001). Random forests. \emph{Machine Learning},
\emph{45}(1), 5--32.

\hypertarget{ref-clark2010mathematical}{}
Clark, C. W. (2010). \emph{Mathematical bioeconomics: The mathematics of
conservation} (Vol. 91). John Wiley \& Sons.

\hypertarget{ref-cope2011reconciling}{}
Cope, J. M., \& Punt, A. E. (2011). Reconciling stock assessment and
management scales under conditions of spatially varying catch histories.
\emph{Fisheries Research}, \emph{107}(1), 22--38.

\hypertarget{ref-cox2002modeling}{}
Cox, S. P., \& Walters, C. (2002). Modeling exploitation in recreational
fisheries and implications for effort management on british columbia
rainbow trout lakes. \emph{North American Journal of Fisheries
Management}, \emph{22}(1), 21--34.

\hypertarget{ref-cox2009evaluation}{}
Cox, S., \& Kronlund, A. (2009). Evaluation of interim harvest
strategies for sablefish (anoplopoma fimbria) in british columbia,
canada for 2008/09. \emph{DFO Can. Sci. Advis. Sec. Res. Doc},
\emph{42}.

\hypertarget{ref-cox2011management}{}
Cox, S., Kronlund, A., \& Lacko, L. (2011). Management procedures for
the multi-gear sablefish (anoplopoma fimbria) fishery in british
columbia, canada. \emph{Can. Sci. Advis. Secret. Res. Doc}, \emph{62}.

\hypertarget{ref-crowder1998fisheries}{}
Crowder, L. B., \& Murawski, S. A. (1998). Fisheries bycatch:
Implications for management. \emph{Fisheries}, \emph{23}(6), 8--17.

\hypertarget{ref-de-la-Mare1998Tidier-fisherie}{}
de la Mare, W. K. (1998). Tidier fisheries management requires a new MOP
(management-oriented paradigm). \emph{Reviews in Fish Biology and
Fisheries}, \emph{8}(3), 349--356.

\hypertarget{ref-de2002fitting}{}
de Valpine, P., \& Hastings, A. (2002). Fitting population models
incorporating process noise and observation error. \emph{Ecological
Monographs}, \emph{72}(1), 57--76.

\hypertarget{ref-DFO1999Fraser-River-Ch}{}
DFO. (1999). \emph{Fraser river chinook salmon} (No. D6-11). DFO Science
Stock Status Report.

\hypertarget{ref-dfo2014stock-asse}{}
DFO. (2014). Stock Assessment and Harvest Advice for Rock Sole
\emph{(lepidopsetta spp.)} in British Columbia. \emph{DFO Can. Sci. Adv.
Sec. Sci. Advis. Rep.}, (2014/039).

\hypertarget{ref-dfo2015arrowtooth}{}
DFO. (2015). Arrowtooth Flounder \emph{(atheresthes stomias)} stock
assessment for the west coast of British Columbia. \emph{DFO Can. Sci.
Advis. Sec. Sci. Advis. Rep.}, (2015/055).

\hypertarget{ref-dichmont2006management2}{}
Dichmont, C. M., Deng, A. R., Punt, A. E., Venables, W., \& Haddon, M.
(2006). Management strategies for short lived species: The case of
Australia's Northern prawn fishery: 2. Choosing appropriate management
strategies using input controls. \emph{Fisheries Research},
\emph{82}(1), 221--234.

\hypertarget{ref-Driscoll2014Groundfish-comp}{}
Driscoll, J. (2014). \emph{Groundfish complex, British Columbia}.
Seafood Watch, Monterey Bay Aquarium.

\hypertarget{ref-dulvy2003extinction}{}
Dulvy, N. K., Sadovy, Y., \& Reynolds, J. D. (2003). Extinction
vulnerability in marine populations. \emph{Fish and Fisheries},
\emph{4}(1), 25--64.

\hypertarget{ref-english2011fraser}{}
English, K. K., Edgell, T. C., Bocking, R. C., Link, M. R., \& Raborn,
S. W. (2011). Fraser river Sockeye fisheries and fisheries management
and comparison with Bristol bay Sockeye fisheries. \emph{The Cohen
Commission of Inquiry into the Decline of Sockeye Salmon in the Fraser
River}.

\hypertarget{ref-fargo1999flatfish-s}{}
Fargo, J. (1999). Flatfish stock assessments for the west coast of
Canada for 1999 and recommended yield options for 2000. \emph{DFO Can.
Stock. Assess. Sec. Res. Doc.}, (1999/199), 51.

\hypertarget{ref-fargo2001turbot-sto}{}
Fargo, J., \& Starr, P. J. (2001). Turbot Stock Assessment for 2001 and
Recommendations for Management in 2002. \emph{DFO Can. Sci. Advis. Sec.
Advis. Res. Doc.}, (2001/150), 70.

\hypertarget{ref-fargo2000stock-asse}{}
Fargo, J., Kronlund, A. R., Schnute, J. T., \& Haigh, R. (2000). Stock
assessment of Rock sole and English sole in Hecate Strait for 2000/2001.
\emph{DFO Can. Stock. Assess. Res. Doc.}, (2000/166), 83.

\hypertarget{ref-PRIFMP2015}{}
Fisheries and Oceans, Canada. (2015). Pacific Region Integrated
Fisheries Management Plan: Groundfish.

\hypertarget{ref-FAO1995}{}
Food and Agriculture Organization of the United Nations. (1995). Code of
conduct for responsible fisheries.

\hypertarget{ref-fournier2012ad}{}
Fournier, D. A., Skaug, H. J., Ancheta, J., Ianelli, J., Magnusson, A.,
Maunder, M. N., \ldots{} Sibert, J. (2012). AD model builder: Using
automatic differentiation for statistical inference of highly
parameterized complex nonlinear models. \emph{Optimization Methods and
Software}, \emph{27}(2), 233--249.

\hypertarget{ref-freeman2008comparison}{}
Freeman, E. A., \& Moisen, G. G. (2008). A comparison of the performance
of threshold criteria for binary classification in terms of predicted
prevalence and kappa. \emph{Ecological Modelling}, \emph{217}(1),
48--58.

\hypertarget{ref-fruhwirth2006finite}{}
Frühwirth-Schnatter, S. (2006). \emph{Finite mixture and markov
switching models}. Springer Science \&amp; Business Media.

\hypertarget{ref-gaichas2012assembly}{}
Gaichas, S., Gamble, R., Fogarty, M., Benoît, H., Essington, T., Fu, C.,
\ldots{} Link, J. (2012). Assembly rules for aggregate-species
production models: Simulations in support of management strategy
evaluation. \emph{Marine Ecology Progress Series}, \emph{459}, 275--292.

\hypertarget{ref-gelman2014bayesian}{}
Gelman, A., Carlin, J. B., Stern, H. S., \& Rubin, D. B. (2014).
\emph{Bayesian data analysis} (Vol. 2). Taylor \& Francis.

\hypertarget{ref-gulland1984observed}{}
Gulland, J., \& Garcia, S. (1984). Observed patterns in multispecies
fisheries. In \emph{Exploitation of marine communities} (pp. 155--190).
Springer.

\hypertarget{ref-hart1973pacific}{}
Hart, J. L., Clemens, W. A., \& others. (1973). \emph{Pacific fishes of
canada}. Fisheries Research Board of Canada.

\hypertarget{ref-hastie2009elements}{}
Hastie, T., Tibshirani, R., \& Friedman, J. (2009). \emph{The elements
of statistical learning} (Vol. 2). Springer.

\hypertarget{ref-hilborn1987general}{}
Hilborn, R., \& Walters, C. J. (1987). A general model for simulation of
stock and fleet dynamics in spatially heterogeneous fisheries.
\emph{Canadian Journal of Fisheries and Aquatic Sciences}, \emph{44}(7),
1366--1369.

\hypertarget{ref-hilborn1992quantitative}{}
Hilborn, R., \& Walters, C. J. (1992). \emph{Quantitative fisheries
stock assessment: Choice, dynamics and uncertainty/Book and disk}.
Springer Science \& Business Media.

\hypertarget{ref-hilborn2004beyond}{}
Hilborn, R., Punt, A. E., \& Orensanz, J. (2004). Beyond band-aids in
fisheries management: Fixing world fisheries. \emph{Bulletin of Marine
Science}, \emph{74}(3), 493--507.

\hypertarget{ref-hilborn2003biocomplexity}{}
Hilborn, R., Quinn, T. P., Schindler, D. E., \& Rogers, D. E. (2003).
Biocomplexity and fisheries sustainability. \emph{Proceedings of the
National Academy of Sciences}, \emph{100}(11), 6564--6568.

\hypertarget{ref-holt2016stock-asse}{}
Holt, K. R., Starr, P. J., Haigh, R., \& Krishka, B. (2016). Stock
Assessment and Harvest Advice for Rock Sole (\emph{Lepidopsetta spp.})
in British Columbia. \emph{DFO Can. Sci. Advis. Sec. Res. Doc.},
(2016/009), ix + 256.

\hypertarget{ref-hutchings2012canada}{}
Hutchings, J. A., Côté, I. M., Dodson, J. J., Fleming, I. A., Jennings,
S., Mantua, N. J., \ldots{} VanderZwaag, D. L. (2012). Is Canada
fulfilling its obligations to sustain marine biodiversity? A summary
review, conclusions, and recommendations. \emph{Environmental Reviews},
\emph{20}(4), 353--361.

\hypertarget{ref-jennings1999structural}{}
Jennings, S., Greenstreet, S., \& Reynolds, J. (1999). Structural change
in an exploited fish community: A consequence of differential fishing
effects on species with contrasting life histories. \emph{Journal of
Animal Ecology}, \emph{68}(3), 617--627.

\hypertarget{ref-jiao2009hierarchical}{}
Jiao, Y., Hayes, C., \& Cortés, E. (2009). Hierarchical Bayesian
approach for population dynamics modelling of fish complexes without
species-specific data. \emph{ICES Journal of Marine Science: Journal Du
Conseil}, \emph{66}(2), 367--377.

\hypertarget{ref-jones2009operating}{}
Jones, M. L., Irwin, B. J., Hansen, G. J., Dawson, H. A., Treble, A. J.,
Liu, W., \ldots{} Bence, J. R. (2009). An operating model for the
integrated pest management of great lakes sea lampreys. \emph{The Open
Fish Science Journal}, \emph{2}, 59--73.

\hypertarget{ref-kallianiotis2004fish}{}
Kallianiotis, A., Vidoris, P., \& Sylaios, G. (2004). Fish species
assemblages and geographical sub-areas in the North Aegean Sea, Greece.
\emph{Fisheries Research}, \emph{68}(1), 171--187.

\hypertarget{ref-kell2012robin}{}
Kell, L. T., \& De Bruyn, P. (2012). The Robin Hood approach for data
poor stocks: An example based on Albacore. \emph{Collect. Vol. Sci. Pap.
ICCAT}, \emph{68}(1), 379--386.

\hypertarget{ref-kristensen2015tmb}{}
Kristensen, K., Nielsen, A., Berg, C. W., Skaug, H., \& Bell, B. (2015).
TMB: Automatic differentiation and laplace approximation. \emph{ArXiv
Preprint ArXiv:1509.00660}.

\hypertarget{ref-lennert2007statistical}{}
Lennert-Cody, C. E., \& Berk, R. A. (2007). Statistical learning
procedures for monitoring regulatory compliance: An application to
fisheries data. \emph{Journal of the Royal Statistical Society: Series A
(Statistics in Society)}, \emph{170}(3), 671--689.

\hypertarget{ref-malick2015linking}{}
Malick, M. J., Cox, S. P., Mueter, F. J., Peterman, R. M., \& Bradford,
M. (2015). Linking phytoplankton phenology to salmon productivity along
a North--South gradient in the Northeast Pacific Ocean. \emph{Canadian
Journal of Fisheries and Aquatic Sciences}, \emph{72}(5), 697--708.

\hypertarget{ref-maunder2015use}{}
Maunder, M. N., Deriso, R. B., \& Hanson, C. H. (2015). Use of
state-space population dynamics models in hypothesis testing: Advantages
over simple log-linear regressions for modeling survival, illustrated
with application to longfin smelt (\emph{spirinchus thaleichthys}).
\emph{Fisheries Research}, \emph{164}, 102--111.

\hypertarget{ref-mantyniemi2009value}{}
Mäntyniemi, S., Kuikka, S., Rahikainen, M., Kell, L. T., \& Kaitala, V.
(2009). The value of information in fisheries management: North sea
herring as an example. \emph{ICES Journal of Marine Science: Journal Du
Conseil}, \emph{66}(10), 2278--2283.

\hypertarget{ref-meyer2015e1071}{}
Meyer, D., Dimitriadou, E., Hornik, K., Weingessel, A., \& Leisch, F.
(2015). \emph{E1071: Misc functions of the department of statistics,
probability theory group (formerly: E1071), tU wien}. Retrieved from
\url{https://CRAN.R-project.org/package=e1071}

\hypertarget{ref-mueter2006using}{}
Mueter, F. J., \& Megrey, B. A. (2006). Using multi-species surplus
production models to estimate ecosystem-level maximum sustainable
yields. \emph{Fisheries Research}, \emph{81}(2), 189--201.

\hypertarget{ref-mueter2002opposite}{}
Mueter, F. J., Peterman, R. M., \& Pyper, B. J. (2002). Opposite effects
of ocean temperature on survival rates of 120 stocks of pacific salmon
(\emph{oncorhynchus spp.}) in northern and southern areas.
\emph{Canadian Journal of Fisheries and Aquatic Sciences}, \emph{59}(3),
456--463.

\hypertarget{ref-parkinson2004linking}{}
Parkinson, E. A., Post, J. R., \& Cox, S. P. (2004). Linking the
dynamics of harvest effort to recruitment dynamics in a multistock,
spatially structured fishery. \emph{Canadian Journal of Fisheries and
Aquatic Sciences}, \emph{61}(9), 1658--1670.

\hypertarget{ref-Pelc201556}{}
Pelc, R. A., Max, L. M., Norden, W., Roberts, S., Silverstein, R., \&
Wilding, S. R. (2015). Further action on bycatch could boost united
states fisheries performance. \emph{Marine Policy}, \emph{56}(0),
56--60.
\url{http://doi.org/http://dx.doi.org/10.1016/j.marpol.2015.02.002}

\hypertarget{ref-pestes2008bayesian}{}
Pestes, L. R., Peterman, R. M., Bradford, M. J., \& Wood, C. C. (2008).
Bayesian decision analysis for evaluating management options to promote
recovery of a depleted salmon population. \emph{Conservation Biology},
\emph{22}(2), 351--361.

\hypertarget{ref-peterman1999decision}{}
Peterman, R. M., \& Anderson, J. L. (1999). Decision analysis: A method
for taking uncertainties into account in risk-based decision making.
\emph{Human and Ecological Risk Assessment: An International Journal},
\emph{5}(2), 231--244.

\hypertarget{ref-punt2011among}{}
Punt, A. E., Smith, D. C., \& Smith, A. D. (2011). Among-stock
comparisons for improving stock assessments of data-poor stocks: The
``Robin Hood'' approach. \emph{ICES Journal of Marine Science: Journal
Du Conseil}, \emph{68}(5), 972--981.

\hypertarget{ref-ricker1958maximum}{}
Ricker, W. (1958). Maximum sustained yields from fluctuating
environments and mixed stocks. \emph{Journal of the Fisheries Board of
Canada}, \emph{15}(5), 991--1006.

\hypertarget{ref-ricker1972hereditary}{}
Ricker, W. (1972). Hereditary and environmental factors affecting
certain salmonid populations. \emph{The Stock Concept in Pacific
Salmon}, 19--160.

\hypertarget{ref-ricker1973two}{}
Ricker, W. (1973). Two mechanisms that make it impossible to maintain
peak-period yields from stocks of Pacific salmon and other fishes.
\emph{Journal of the Fisheries Board of Canada}, \emph{30}(9),
1275--1286.

\hypertarget{ref-rokach2010ensemble}{}
Rokach, L. (2010). Ensemble-based classifiers. \emph{Artificial
Intelligence Review}, \emph{33}(1-2), 1--39.

\hypertarget{ref-safina2008study}{}
Safina, C., \& Lewison, R. L. (2008). Why study bycatch? An introduction
to the theme section on fisheries bycatch.

\hypertarget{ref-saila1983fishery}{}
Saila, S. B., \& Jones, C. (1983). \emph{Fishery science and the stock
concept}. University of Rhode Island.

\hypertarget{ref-sainsbury2000design}{}
Sainsbury, K. J., Punt, A. E., \& Smith, A. D. (2000). Design of
operational management strategies for achieving fishery ecosystem
objectives. \emph{ICES Journal of Marine Science: Journal Du Conseil},
\emph{57}(3), 731--741.

\hypertarget{ref-schindler2010population}{}
Schindler, D. E., Hilborn, R., Chasco, B., Boatright, C. P., Quinn, T.
P., Rogers, L. A., \& Webster, M. S. (2010). Population diversity and
the portfolio effect in an exploited species. \emph{Nature},
\emph{465}(7298), 609--612.

\hypertarget{ref-simon1972stock}{}
Simon, R. C., \& Larkin, P. A. (1972). \emph{The stock concept in
pacific salmon}. University of British Columbia.

\hypertarget{ref-starr2009english-so}{}
Starr, P. J. (2009a). English Sole (\emph{Parophrys vetulus}) in British
Columbia, Canada: Stock Assessment for 2006/07 and Advice to Managers
for 2007/08. \emph{DFO Can. Sci. Advis. Sec. Res. Doc.}, (2009/069), v +
149.

\hypertarget{ref-starr2009petrale-so}{}
Starr, P. J. (2009b). Petrale sole (\emph{eopsetta jordani}) in British
Columbia, Canada: Stock Assessment for 2006/07 and Advice to Managers
for 2007/08. \emph{DFO Can. Sci. Advis. Sec. Res. Doc.}, (2009/070), v +
134.

\hypertarget{ref-su2004spatial}{}
Su, Z., Peterman, R. M., \& Haeseker, S. L. (2004). Spatial hierarchical
bayesian models for stock-recruitment analysis of pink salmon
(oncorhynchus gorbuscha). \emph{Canadian Journal of Fisheries and
Aquatic Sciences}, \emph{61}(12), 2471--2486.

\hypertarget{ref-sugihara1984ecosystems}{}
Sugihara, G., Garcia, S., Platt, T., Gulland, J., Rachor, E., Lawton,
J., \ldots{} others. (1984). Ecosystems dynamics. In \emph{Exploitation
of marine communities} (pp. 131--153). Springer.

\hypertarget{ref-team2009national}{}
Team, C. S. R. (2009). \emph{National conservation strategy for cultus
lake sockeye salmon (oncorhynchus nerka)}. Fisheries; Oceans Canada.

\hypertarget{ref-venables2002modern}{}
Venables, W. N., \& Ripley, B. D. (2002). \emph{Modern applied
statistics with s} (Fourth). New York: Springer. Retrieved from
\url{http://www.stats.ox.ac.uk/pub/MASS4}

\hypertarget{ref-vrbik2015fractionally}{}
Vrbik, I., \& McNicholas, P. D. (2015). Fractionally-supervised
classification. \emph{Journal of Classification}, \emph{32}(3),
359--381.

\hypertarget{ref-walters1999multispecies}{}
Walters, C. J., \& Bonfil, R. (1999). Multispecies spatial assessment
models for the British Columbia groundfish trawl fishery. \emph{Canadian
Journal of Fisheries and Aquatic Sciences}, \emph{56}(4), 601--628.

\hypertarget{ref-walters2004fisheries}{}
Walters, C. J., \& Martell, S. J. (2004). \emph{Fisheries ecology and
management}. Princeton University Press.

\hypertarget{ref-zhou2010modified}{}
Zhou, S., Punt, A. E., Deng, R., Dichmont, C. M., Ye, Y., \& Bishop, J.
(2010). Modified hierarchical Bayesian biomass dynamics models for
assessment of short-lived invertebrates: A comparison for tropical tiger
prawns. \emph{Marine and Freshwater Research}, \emph{60}(12),
1298--1308.

\end{document}